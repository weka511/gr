\documentclass[]{article}
\usepackage{esvect,mathtools,amssymb,amsthm,url,tikz}
\newcommand\numberthis{\addtocounter{equation}{1}\tag{\theequation}}
\usepackage[nottoc,numbib]{tocbibind}
\newtheorem{lemma}{Lemma}
%opening
\title{Introduction into General Relativity\\Assignment 10.1\\Gravitational Radiation}
\author{Simon Crase}

\begin{document}

\maketitle

\begin{abstract}
	
	Calculate the average over all directions of $\vec{r}$: $\left\langle \big(\vec{r}.\vec{a}\big)\vec{r} \right\rangle$, where $\vec{a}$ is a fixed vector and $|\vec{r}|=R$ is also fixed.
	
	We present two derivations. Section \ref{sect:long} is in the spirit of the assignment: \emph{"The first problem is just a test of how well one knows the methods of averaging over directions. This is a standard tool which is used in many areas of classical electrodynamics and general relativity: e.g. in the context of the radiation problem and in the context of particle motion in inhomogeneous electromagnetic fields."}. Section \ref{sect:symmetry} exploits symmetries in the problem to give a more concise derivation.
\end{abstract}

\tableofcontents

\section{Calculation of $\left\langle \big(\vec{r}.\vec{a}\big)\vec{r} \right\rangle$ by a multiple integral}\label{sect:long}
\subsection{Preliminary Lemmata}
\begin{lemma}\label{lemma:i1}
	$\int_{0}^{2\pi}(\cos\phi)^2  d\phi = \int_{0}^{2\pi}(\sin\phi)^2  d\phi=\pi$
\end{lemma}
\begin{proof}
	That $\int_{0}^{2\pi}(\cos\phi)^2  d\phi = \int_{0}^{2\pi}(\sin\phi)^2  d\phi$ follows from changing the variable of integration, $\phi\rightarrow\phi+\frac{\pi}{2}$, and exploiting the periodicity of $\sin$ and $\cos$. 
	
	Moreover $\int_{0}^{2\pi}\big[(\cos\phi)^2  +(\sin\phi)^2\big] d\phi=\int_{0}^{2\pi}1. d\phi=2\pi$. Since the two integrals are equal and sum to $2\pi$, each must be $\pi$.
\end{proof}

\begin{lemma}\label{lemma:linearity}
	$\left\langle \big(\vec{r}.\vec{a}\big)\vec{r} \right\rangle$ is a linear transformation of $\vec{a}$
\end{lemma}
\begin{proof}
\begin{align*}
\vec{r}.\vec{a} \triangleq& \sum_{i=1}^{3}r_ia_i\numberthis\label{eq:ra}\\
\big(\vec{r}.\vec{a}\big)\vec{r} =& \sum_{i=1}^{3}r_ia_i \vec{r}\\
=& \sum_{i=1}^{3}r_ia_i \sum_{j=1}^{3}r_j\vec{i}_j \text{, where $\vec{i}_j$ represents the 3 vectors $\vec{i},\vec{j},\vec{k}$}\\
=& \sum_{i,j=1}^{3}r_i  r_j a_i \vec{i}_j \\
\left\langle \big(\vec{r}.\vec{a}\big)\vec{r} \right\rangle=&\left\langle \sum_{i,j=1}^{3}r_i  r_j a_i \vec{i}_j\right\rangle\\
=&\left\langle \sum_{i,j=1}^{3}r_i  r_j\right\rangle a_i \vec{i}_j\\
=&\mathfrak{R} \vec{a}\text{, where } \mathfrak{R}_{ij}=\left\langle \sum_{i,j=1}^{3}r_i  r_j\right\rangle \numberthis\label{eq:rar}
\end{align*}

Equation (\ref{eq:rar}) shows that $\left\langle \big(\vec{r}.\vec{a}\big)\vec{r} \right\rangle$ is indeed a linear transformation of $\vec{a}$.
\end{proof}

\subsection{Derivation}
Since $|r|=R$, a constant, we will use polar coordinates, following the \emph{physics convention}, \cite{schaum-basic} \& \cite{wiki-polar}, and write:
$$\vec{r}=R\sin\theta\cos\phi\vec{i}+R\sin\theta\sin\phi\vec{j}+R\cos\theta\vec{k}$$
so (\ref{eq:ra}) and (\ref{eq:rar}) become:
\begin{align*}
\big(\vec{r}.\vec{a}\big)\vec{r} =& (\vec{r}.\vec{a}) R \sin\theta\cos\phi .\vec{i} + (\vec{r}.\vec{a})R\sin\theta\sin\phi . \vec{j} + (\vec{r}.\vec{a}) R\cos\theta. \vec{k}\\
\vec{r}.\vec{a} =& Ra_1\sin\theta\cos\phi + Ra_2\sin\theta\sin\phi + Ra_3 \cos\theta\\
\text{hence}&\\
\big(\vec{r}.\vec{a}\big)\vec{r} = R^2 \bigg\{& \big[a_1\sin\theta\cos\phi + a_2\sin\theta\sin\phi + a_3 \cos\theta\big] \sin\theta\cos\phi.\vec{i}\\ +& \big[a_1\sin\theta\cos\phi + a_2\sin\theta\sin\phi + a_3 \cos\theta\big]\sin\theta\sin\phi. \vec{j}\\
 +& \big[a_1\sin\theta\cos\phi + a_2\sin\theta\sin\phi + a_3 \cos\theta\big] \cos\theta. \vec{k}\bigg\}\numberthis \label{eq:rar_expanded}
\end{align*}
Now, by definition
\begin{align*}
\left\langle \big(\vec{r}.\vec{a}\big)\vec{r} \right\rangle \triangleq& \frac{1}{4\pi}\int_{\Omega}\big(\vec{r}.\vec{a}\big)\vec{r} d\Omega \\
=&  \frac{ R^2}{4\pi}\int_{\Omega} \bigg\{ \big[a_1\sin\theta\cos\phi + a_2\sin\theta\sin\phi + a_3 \sin\theta\big] \sin\theta\cos\phi.\vec{i}\\ +& \big[a_1\sin\theta\cos\phi + a_2\sin\theta\sin\phi + a_3 \cos\theta\big]\sin\theta\sin\phi .\vec{j}\\
+& \big[a_1\sin\theta\cos\phi + a_2\sin\theta\sin\phi + a_3 \cos\theta\big] \cos\theta. \vec{k}\bigg\}d\Omega\\
= \frac{ R^2}{4\pi} \bigg\{&\int_{\Omega}\big[a_1\sin\theta\cos\phi + a_2\sin\theta\sin\phi + a_3 \cos\theta\big] \sin\theta\cos\phi.\vec{i}d\Omega\\ +& \int_{\Omega}\big[a_1\sin\theta\cos\phi + a_2\sin\theta\sin\phi + a_3 \cos\theta\big]\sin\theta\sin\phi .\vec{j}d\Omega\\
+& \int_{\Omega}\big[a_1\sin\theta\cos\phi + a_2\sin\theta\sin\phi + a_3 \cos\theta\big] \cos\theta. \vec{k} d\Omega\bigg\}\\
= \frac{ R^2}{4\pi} \bigg\{&\int_{\theta=0}^{\pi}\int_{\phi=0}^{2\pi}\big[a_1\sin\theta\cos\phi + a_2\sin\theta\sin\phi + a_3 \cos\theta\big] \sin\theta\cos\phi\sin\theta d\theta d\phi.\vec{i}\\
 +& \int_{\theta=0}^{\pi}\int_{\phi=0}^{2\pi}\big[a_1\sin\theta\cos\phi + a_2\sin\theta\sin\phi + a_3 \cos\theta\big]\sin\theta\sin\phi \sin\theta d\theta d\phi.\vec{j}\\
+& \int_{\theta=0}^{\pi}\int_{\phi=0}^{2\pi}\big[a_1\sin\theta\cos\phi + a_2\sin\theta\sin\phi + a_3 \cos\theta\big] \cos\theta \sin\theta d\theta d\phi.\vec{k}\\ \bigg\} \numberthis \label{eq:big-int}
\end{align*}

If we extract products such as $a_1.\vec{i}$, we find nine integrals as follows.
\begin{align*}
\text{Coefficient of }a_1.\vec{i} =&\int_{\theta=0}^{\pi}\int_{\phi=0}^{2\pi}(\sin\theta)^3(\cos\phi)^2 d\theta d\phi\\
=&\int_{\theta=0}^{\pi}(\sin\theta)^3d\theta\int_{\phi=0}^{2\pi}(\cos\phi)^2  d\phi\\
=&\pi \int_{0}^{\pi}(\sin\theta)^3d\theta\text{, from Lemma \ref{lemma:i1}}\\
=&\pi \int_{0}^{\pi} [1-(\cos \theta)^2] \sin\theta d\theta  \\
=& \pi [-\cos\theta+\frac{1}{3}(\cos\theta)^3]\Bigg|_0^{\pi} \\
=& \pi [1-(-1)-\frac{1} {3}-\frac{1} {3}] \\
=& \frac{4\pi}{3}\numberthis \label{eq:a1-i}
\end{align*}

\begin{align*}
\text{Coefficient of }a_2.\vec{i} =&\int_{\theta=0}^{\pi}\int_{\phi=0}^{2\pi}(\sin\theta)^3\sin\phi\cos\phi\ d\theta d\phi\\=&\int_{\theta=0}^{\pi}(\sin\theta)^3d\theta\int_{\phi=0}^{2\pi}\sin\phi\cos\phi\  d\phi\\
=&\int_{\theta=0}^{\pi}(\sin\theta)^3d\theta . \frac{1}{2} (\sin \phi)^2 \Big|_0^{2\pi} \\
=&\int_{\theta=0}^{\pi}(\sin\theta)^3d\theta\times 0\\
=&0 \numberthis \label{eq:a2-i}
\end{align*}

\begin{align*}
\text{Coefficient of }a_3.\vec{i} =&\int_{\theta=0}^{\pi}\int_{\phi=0}^{2\pi}\cos\theta(\sin\theta)^2\cos\phi d\theta d\phi\\=&\int_{\theta=0}^{\pi}\cos\theta(\sin\theta)^2d\theta\int_{\phi=0}^{2\pi}\cos\phi  d\phi\\
=&\int_{\theta=0}^{\pi}\cos\theta(\sin\theta)^2d\theta . \sin\phi \Big|_0^{2\pi}\\
=&\int_{\theta=0}^{\pi}\cos\theta(\sin\theta)^2d\theta \times 0\\
=&0 \numberthis \label{eq:a3-i}
\end{align*}

\begin{align*}
\text{Coefficient of }a_1.\vec{j} =&\int_{\theta=0}^{\pi}\int_{\phi=0}^{2\pi}(\sin\theta)^3\cos\phi\sin\phi d\theta d\phi\\=&\int_{\theta=0}^{\pi}(\sin\theta)^3d\theta\int_{\phi=0}^{2\pi}\cos\phi\sin\phi  d\phi\\
=&0 \text{, by a similar argument to (\ref{eq:a2-i})}\numberthis \label{eq:a1-j}
\end{align*}

\begin{align*}
\text{Coefficient of }a_2.\vec{j} =&\int_{\theta=0}^{\pi}\int_{\phi=0}^{2\pi}(\sin\theta)^3(\sin\phi)^2 d\theta d\phi\\=&\int_{\theta=0}^{\pi}(\sin\theta)^3 d\theta \int_{\phi=0}^{2\pi}(\sin\phi)^2  d\phi\\
=& \frac{4\pi}{3} \text{, by a similar argument to (\ref{eq:a1-i})} \numberthis \label{eq:a2-j}
\end{align*}

\begin{align*}
\text{Coefficient of }a_3.\vec{j} =&\int_{\theta=0}^{\pi}\int_{\phi=0}^{2\pi}\cos\theta(\sin\theta)^2\sin\phi  d\theta d\phi\\=&\int_{\theta=0}^{\pi}\cos\theta(\sin\theta)^2d\theta\int_{\phi=0}^{2\pi}\sin\phi   d\phi\\=&0 \text{, by a similar argument to (\ref{eq:a3-i})} \numberthis \label{eq:a3-j}
\end{align*}

\begin{align*}
\text{Coefficient of }a_1.\vec{k} =&\int_{\theta=0}^{\pi}\int_{\phi=0}^{2\pi}(\sin\theta)^2\cos\phi\cos\theta d\theta d\phi\\=&
\int_{\theta=0}^{\pi}  (\sin\theta)^2 cos\theta d\theta \int_{\phi=0}^{2\pi}\cos\phi\ d\phi\\
=&0 \text{, by a similar argument to (\ref{eq:a3-i})} \numberthis \label{eq:a1-k}
\end{align*}

\begin{align*}
\text{Coefficient of }a_2.\vec{k} =&\int_{\theta=0}^{\pi}\int_{\phi=0}^{2\pi}(\sin\theta)^2\sin\phi\cos\theta d\theta d\phi\\=&
\int_{\theta=0}^{\pi} (\sin\theta)^2 \cos\theta d\theta \int_{\phi=0}^{2\pi}\sin\phi d\phi\\=&0 \text{, by a similar argument to (\ref{eq:a2-i})} \numberthis \label{eq:a2-k}
\end{align*}

\begin{align*}
\text{Coefficient of }a_3.\vec{k} =&\int_{\theta=0}^{\pi} \int_{\phi=0}^{2\pi} (\cos\theta)^2\sin\theta d\theta   d\phi\\ &= \int_{\theta=0}^{\pi} (\cos\theta)^2\sin\theta d\theta  \int_{\phi=0}^{2\pi} d\phi \\ &= 2 \pi \int_{\theta=0}^{\pi} (\cos\theta)^2\sin\theta d\theta \\
=& 2\pi[-\frac{1}{3}(\cos\theta)^3] \Bigg|_0^{\pi} \\
=& \frac{4\pi}{3}\numberthis \label{eq:a3-k}
\end{align*}

Substituting (\ref{eq:a1-i}--\ref{eq:a3-k})  in (\ref{eq:big-int})
\begin{align*}
\left\langle \big(\vec{r}.\vec{a}\big)\vec{r} \right\rangle=&\frac{ R^2}{4\pi}\frac{4\pi}{3}\big[a_1.\vec{i}+a_2.\vec{j}+a_3.\vec{k}\big]\\
=&\frac{R^2}{3}\vec{a} \numberthis \label{eq:final}
\end{align*}
Hence the linear transformation in Lemma \ref{lemma:linearity}  takes a simple form:
\[\mathfrak{R}= \frac{R^2}{3}
\begin{pmatrix}
1 & 0 & 0 \\
0 & 1 & 0 \\
0 & 0 & 1
\end{pmatrix}
\]

\section{Calculation by a symmetry argument}\label{sect:symmetry}
The simple form of (\ref{eq:final}) suggest that a simpler derivation may be possible using symmetry. From (\ref{eq:rar}), $\mathfrak{R}$ is a real symmetric matrix. By the \emph{spectral theorem}, \cite{spectral-theorem} \& \cite{bellman1970}, $\mathfrak{R}$ has real eigenvalues and we can find 3 orthonormal eigenvectors\footnote{If the eigenvalues are distinct, the eigenvectors are unique to within a scaler constant; we shall show that they are degerate, so the bases is not unique.}. If we adopt a set of 3 orthonormal eigenvectors as our basis, $\mathfrak{R}$ becomes a diagonal matrix, so we can write:
\[\mathfrak{R}= 
\begin{pmatrix}
R_1 & 0 & 0 \\
0 & R_2 & 0 \\
0 & 0 & R_3
\end{pmatrix}
\]
We assert that the three diagonal elements (the eigenvalues) are equal, hence $\mathfrak{R}$ is diagonal in any orthonormal basis. For otherwise at least two of the diagonal elements must be unequal. Without loss of generality we can suppose that $R_1 \neq R_2$. Hence if $\vec{a}$ were parallel to the 1-axis it would be stretched by a different amount compared to if it were parallel to the 2-axis; we could exploit this to \emph{define a privileged direction in space,} which minimizes the length of $\mathfrak{R}\vec{a}$. This violates the assumption that space is isotropic, hence there is no privileged definition and $R_1=R_2=R_3$; $\mathfrak{R}\vec{a}=\rho\vec{a}$, where $\rho$ represents the (degenerate) single eigenvalue of $\mathfrak{R}$. Inspection of (\ref{eq:rar}) shows that $\rho \propto R^2$, so we can write $\big(\vec{r}.\vec{a}\big)\vec{r}=\kappa R^2 \vec{a}$, where $\kappa$ is a constant of proportionately. Compare with (\ref{eq:final}).

The foregoing allows us to dispense with all but one of equations (\ref{eq:a1-i}--\ref{eq:a3-k}). For example, we could use (\ref{eq:a1-i}) to show that $\kappa=\frac{1}{3}$, and discard the remaining 8 equations.

\begin{lemma}
	$\kappa=\frac{1}{3}$
\end{lemma}
\begin{proof}
	The foregoing argument shows that we can choose $\vec{a}$ to simplify our calculations, since space is isotropic. We shall therefore choose $\vec{a}$ to be a unit vector at the "North Pole". Let $\vec{r}$ be an arbitrary vector that makes an angle $\theta$ with the North Pole. The symmetry of the configuration allows us to divide the surface of the sphere into annuli, centred on the North Pole, and perform the calculation as a single integral.
	\begin{align*}
	\big(\vec{r}.\vec{a}\big)\vec{r}\Bigg|_\text{projected onto z}=&\frac{1}{4\pi}\int_{0}^{\pi}\underbrace{|\vec{a}||\vec{r}|\cos\theta}_{\vec{r}.\vec{a}}\overbrace{R \cos \theta}^{\text{Projection of $\vec{r}$ onto z}}\underbrace{2\pi \sin\theta d\theta}_\text{Area of annulus}\\
	=& \frac{R^2|\vec{a}|}{2}\int_{0}^{2\pi}\cos^2 \theta \sin \theta d \theta\\
	=& - \frac{R^2|\vec{a}|}{6} cos^3\theta\Bigg |_0^{\pi}\\
	=& - \frac{R^2|\vec{a}|}{6} [-1-1]\\
	=& \frac{R^2|\vec{a}|}{3} \numberthis \label{eq:kappa}
	\end{align*}
\end{proof}
Compare equation (\ref{eq:kappa}) with (\ref{eq:a3-k}).
 
\begin{thebibliography}{9}\label{section:biblio}
	\raggedright
	\bibitem{schaum-basic}
	William F. Hughes \& Eber W. Gaylord,
	\emph{Basic Equations of Engineering Science},
	Schaum Publishing Company, New York,
	ISBN-13: 978-0070311091
	1964
	\bibitem{wiki-polar}
	Wikipedia contributors. Spherical coordinate system. Wikipedia, The Free Encyclopedia. August 19, 2017, 21:40 UTC. Available at: 
	\url{https://en.wikipedia.org/w/index.php?title=Spherical_coordinate_system&oldid=796300234}.
	Accessed
	October 13, 2017
	\bibitem{spectral-theorem}
	Daniel Jerison,
	\emph{Math 2940: Symmetric matrices have real eigenvalues},
	\url{http://www.math.cornell.edu/~jerison/math2940/real-eigenvalues.pdf},
	Accessed 14 October 2017
	\bibitem{bellman1970}
	Richard Bellman,
	\emph{Introduction to Matrix Analysis},
	1960, republished 1997,
	ISBN	0898713994, 9780898713992
\end{thebibliography}

\end{document}
