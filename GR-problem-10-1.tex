\documentclass[]{article}
\usepackage{esvect,mathtools,amssymb,amsthm,url}
\newcommand\numberthis{\addtocounter{equation}{1}\tag{\theequation}}
%opening
\title{Introduction into General Relativity\\Assignment 10.1\\Gravitational Radiation}
\author{Simon Crase}

\begin{document}

\maketitle

\begin{abstract}
Calculate the average over all directions of $\vec{r}$:

$$\left\langle \big(\vec{r}.\vec{a}\big)\vec{r} \right\rangle$$

where $\vec{a}$ is a fixed vector and $|\vec{r}|=R$ is also fixed.
\begin{enumerate}
	\item correct calculation of the vector components of the average
	\item the presence of the final answer in the vector form
	

\end{enumerate}
\end{abstract}

\begin{align*}
\vec{r}.\vec{a} \triangleq& \sum_{i=1}^{3}r_ia_i\numberthis\label{eq:ra}\\
\big(\vec{r}.\vec{a}\big)\vec{r} =& \sum_{i=1}^{3}r_ia_i \vec{r} \numberthis\label{eq:rar}
\end{align*}

From (\ref{eq:rar}) it is clear that the desired expression is a linear transformation of $\vec{a}$, i.e. $\exists \mathfrak{R}$, a matrix, such that $\big(\vec{r}.\vec{a}\big)\vec{r} =\mathfrak{R}\vec{a}$ 

Since $|r|=R$, we will use polar coordinates and write:
$$\vec{r}=R\sin\theta\cos\phi\vec{i}+R\sin\theta\sin\phi\vec{j}+R\cos\theta\vec{k}$$
so (\ref{eq:ra}) and (\ref{eq:rar})become:
\begin{align*}
\big(\vec{r}.\vec{a}\big)\vec{r} =& (\vec{r}.\vec{a}) R \sin\theta\cos\phi\ \vec{i} + (\vec{r}.\vec{a})R\sin\theta\sin\phi  \vec{j} + (\vec{r}.\vec{a}) R\cos\theta \vec{k}\\
\vec{r}.\vec{a} =& Ra_1\sin\theta\cos\phi + Ra_2\sin\theta\sin\phi + Ra_3 \cos\theta\\
\text{hence}&\\
\big(\vec{r}.\vec{a}\big)\vec{r} = R^2 \bigg\{& \big[a_1\sin\theta\cos\phi + a_2\sin\theta\sin\phi + a_3 \cos\theta\big] \sin\theta\cos\phi\vec{i}\\ +& \big[a_1\sin\theta\cos\phi + a_2\sin\theta\sin\phi + a_3 \cos\theta\big]\sin\theta\sin\phi \vec{j}\\
 +& \big[a_1\sin\theta\cos\phi + a_2\sin\theta\sin\phi + a_3 \cos\theta\big] \cos\theta \vec{k}\bigg\}\numberthis \label{eq:rar_expanded}
\end{align*}
Now, by definition
\begin{align*}
\left\langle \big(\vec{r}.\vec{a}\big)\vec{r} \right\rangle \triangleq& \frac{1}{4\pi}\int_{\Omega}\big(\vec{r}.\vec{a}\big)\vec{r} d\Omega \\
=&  \frac{ R^2}{4\pi}\int_{\Omega} \bigg\{ \big[a_1\sin\theta\cos\phi + a_2\sin\theta\sin\phi + a_3 \sin\theta\big] \sin\theta\cos\phi\vec{i}\\ +& \big[a_1\sin\theta\cos\phi + a_2\sin\theta\sin\phi + a_3 \cos\theta\big]\sin\theta\sin\phi \vec{j}\\
+& \big[a_1\sin\theta\cos\phi + a_2\sin\theta\sin\phi + a_3 \cos\theta\big] \cos\theta \vec{k}\bigg\}d\Omega\\
= \frac{ R^2}{4\pi} \bigg\{&\int_{\Omega}\big[a_1\sin\theta\cos\phi + a_2\sin\theta\sin\phi + a_3 \cos\theta\big] \sin\theta\cos\phi\vec{i}d\Omega\\ +& \int_{\Omega}\big[a_1\sin\theta\cos\phi + a_2\sin\theta\sin\phi + a_3 \cos\theta\big]\sin\theta\sin\phi \vec{j}d\Omega\\
+& \int_{\Omega}\big[a_1\sin\theta\cos\phi + a_2\sin\theta\sin\phi + a_3 \cos\theta\big] \cos\theta \vec{k} d\Omega\bigg\}\\
= \frac{ R^2}{4\pi} \bigg\{&\int_{\theta=-\pi}^{\pi}\int_{\phi=0}^{2\pi}\big[a_1\sin\theta\cos\phi + a_2\sin\theta\sin\phi + a_3 \cos\theta\big] \sin\theta\cos\phi\sin\theta d\theta d\phi.\vec{i}\\
 +& \int_{\theta=-\pi}^{\pi}\int_{\phi=0}^{2\pi}\big[a_1\sin\theta\cos\phi + a_2\sin\theta\sin\phi + a_3 \cos\theta\big]\sin\theta\sin\phi \sin\theta d\theta d\phi.\vec{j}\\
+& \int_{\theta=-\pi}^{\pi}\int_{\phi=0}^{2\pi}\big[a_1\sin\theta\cos\phi + a_2\sin\theta\sin\phi + a_3 \cos\theta\big] \cos\theta \sin\theta d\theta d\phi.\vec{k}\\ \bigg\} \numberthis \label{eq:big-int}
\end{align*}

\begin{align*}
\int_{\theta=-\pi}^{\pi}\int_{\phi=0}^{2\pi}\sin\theta\cos\phi\sin\theta\cos\phi\sin\theta d\theta d\phi
\end{align*}

\begin{align*}
\int_{\theta=-\pi}^{\pi}\int_{\phi=0}^{2\pi}\sin\theta\sin\phi\sin\theta\cos\phi\sin\theta d\theta d\phi
\end{align*}

\begin{align*}
\int_{\theta=-\pi}^{\pi}\int_{\phi=0}^{2\pi}\cos\theta\sin\theta\cos\phi\sin\theta d\theta d\phi
\end{align*}

\begin{align*}
\int_{\theta=-\pi}^{\pi}\int_{\phi=0}^{2\pi}\sin\theta\cos\phi\sin\theta\sin\phi \sin\theta d\theta d\phi
\end{align*}

\begin{align*}
\int_{\theta=-\pi}^{\pi}\int_{\phi=0}^{2\pi}\sin\theta\sin\phi\sin\theta\sin\phi \sin\theta d\theta d\phi
\end{align*}

\begin{align*}
\int_{\theta=-\pi}^{\pi}\int_{\phi=0}^{2\pi}\cos\theta\sin\theta\sin\phi \sin\theta d\theta d\phi
\end{align*}

\begin{align*}
\int_{\theta=-\pi}^{\pi}\int_{\phi=0}^{2\pi}\sin\theta\cos\phi\cos\theta\sin\theta d\theta d\phi
\end{align*}

\begin{align*}
\int_{\theta=-\pi}^{\pi}\int_{\phi=0}^{2\pi}\sin\theta\sin\phi\cos\theta\sin\theta d\theta d\phi
\end{align*}

\begin{align*}
\int_{\theta=-\pi}^{\pi}\int_{\phi=0}^{2\pi}\cos\theta\cos\theta\sin\theta d\theta d\phi
\end{align*}

\end{document}
