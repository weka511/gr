\documentclass[]{article}
\usepackage{esvect,mathtools,amssymb,amsthm,url}
\newcommand\numberthis{\addtocounter{equation}{1}\tag{\theequation}}
\newtheorem{lemma}{Lemma}
%opening
\title{Introduction into General Relativity\\Assignment 10.1\\Gravitational Radiation}
\author{Simon Crase}

\begin{document}

\maketitle

Calculate the average over all directions of $\vec{r}$:

$$\left\langle \big(\vec{r}.\vec{a}\big)\vec{r} \right\rangle$$

where $\vec{a}$ is a fixed vector and $|\vec{r}|=R$ is also fixed.

\begin{lemma}\label{lemma:i1}
	$\int_{0}^{2\pi}(\cos\phi)^2  d\phi = \int_{0}^{2\pi}(\sin\phi)^2  d\phi=\pi$
\end{lemma}
\begin{proof}
	$\int_{0}^{2\pi}(\cos\phi)^2  d\phi = \int_{0}^{2\pi}(\sin\phi)^2  d\phi$ follows from a change of variable $\phi\rightarrow\phi+\frac{\pi}{2}$, and exploiting the periodicity of $\sin$ and $\cos$. 
	
	Moreover $\int_{0}^{2\pi}\big[(\cos\phi)^2  +(\sin\phi)^2\big] d\phi=\int_{0}^{2\pi}1. d\phi=2\pi$. Since the two integrals are equal and add to $2\pi$, each must be $\pi$.
\end{proof}
\begin{align*}
\vec{r}.\vec{a} \triangleq& \sum_{i=1}^{3}r_ia_i\numberthis\label{eq:ra}\\
\big(\vec{r}.\vec{a}\big)\vec{r} =& \sum_{i=1}^{3}r_ia_i \vec{r} \numberthis\label{eq:rar}
\end{align*}

From (\ref{eq:rar}) it is clear that the desired expression is a linear transformation of $\vec{a}$, i.e. $\exists \mathfrak{R}$, a matrix, such that:
\begin{align*}
\big(\vec{r}.\vec{a}\big)\vec{r} =\mathfrak{R}\vec{a} \numberthis \label{eq:linear}
\end{align*} 

Since $|r|=R$, a constant, we will use polar coordinates, following the \emph{physics convention}\cite{wiki-polar}, and write:
$$\vec{r}=R\sin\theta\cos\phi\vec{i}+R\sin\theta\sin\phi\vec{j}+R\cos\theta\vec{k}$$
so (\ref{eq:ra}) and (\ref{eq:rar}) become:
\begin{align*}
\big(\vec{r}.\vec{a}\big)\vec{r} =& (\vec{r}.\vec{a}) R \sin\theta\cos\phi .\vec{i} + (\vec{r}.\vec{a})R\sin\theta\sin\phi . \vec{j} + (\vec{r}.\vec{a}) R\cos\theta. \vec{k}\\
\vec{r}.\vec{a} =& Ra_1\sin\theta\cos\phi + Ra_2\sin\theta\sin\phi + Ra_3 \cos\theta\\
\text{hence}&\\
\big(\vec{r}.\vec{a}\big)\vec{r} = R^2 \bigg\{& \big[a_1\sin\theta\cos\phi + a_2\sin\theta\sin\phi + a_3 \cos\theta\big] \sin\theta\cos\phi.\vec{i}\\ +& \big[a_1\sin\theta\cos\phi + a_2\sin\theta\sin\phi + a_3 \cos\theta\big]\sin\theta\sin\phi. \vec{j}\\
 +& \big[a_1\sin\theta\cos\phi + a_2\sin\theta\sin\phi + a_3 \cos\theta\big] \cos\theta. \vec{k}\bigg\}\numberthis \label{eq:rar_expanded}
\end{align*}
Now, by definition
\begin{align*}
\left\langle \big(\vec{r}.\vec{a}\big)\vec{r} \right\rangle \triangleq& \frac{1}{4\pi}\int_{\Omega}\big(\vec{r}.\vec{a}\big)\vec{r} d\Omega \\
=&  \frac{ R^2}{4\pi}\int_{\Omega} \bigg\{ \big[a_1\sin\theta\cos\phi + a_2\sin\theta\sin\phi + a_3 \sin\theta\big] \sin\theta\cos\phi.\vec{i}\\ +& \big[a_1\sin\theta\cos\phi + a_2\sin\theta\sin\phi + a_3 \cos\theta\big]\sin\theta\sin\phi .\vec{j}\\
+& \big[a_1\sin\theta\cos\phi + a_2\sin\theta\sin\phi + a_3 \cos\theta\big] \cos\theta. \vec{k}\bigg\}d\Omega\\
= \frac{ R^2}{4\pi} \bigg\{&\int_{\Omega}\big[a_1\sin\theta\cos\phi + a_2\sin\theta\sin\phi + a_3 \cos\theta\big] \sin\theta\cos\phi.\vec{i}d\Omega\\ +& \int_{\Omega}\big[a_1\sin\theta\cos\phi + a_2\sin\theta\sin\phi + a_3 \cos\theta\big]\sin\theta\sin\phi .\vec{j}d\Omega\\
+& \int_{\Omega}\big[a_1\sin\theta\cos\phi + a_2\sin\theta\sin\phi + a_3 \cos\theta\big] \cos\theta. \vec{k} d\Omega\bigg\}\\
= \frac{ R^2}{4\pi} \bigg\{&\int_{\theta=0}^{\pi}\int_{\phi=0}^{2\pi}\big[a_1\sin\theta\cos\phi + a_2\sin\theta\sin\phi + a_3 \cos\theta\big] \sin\theta\cos\phi\sin\theta d\theta d\phi.\vec{i}\\
 +& \int_{\theta=0}^{\pi}\int_{\phi=0}^{2\pi}\big[a_1\sin\theta\cos\phi + a_2\sin\theta\sin\phi + a_3 \cos\theta\big]\sin\theta\sin\phi \sin\theta d\theta d\phi.\vec{j}\\
+& \int_{\theta=0}^{\pi}\int_{\phi=0}^{2\pi}\big[a_1\sin\theta\cos\phi + a_2\sin\theta\sin\phi + a_3 \cos\theta\big] \cos\theta \sin\theta d\theta d\phi.\vec{k}\\ \bigg\} \numberthis \label{eq:big-int}
\end{align*}

If we extract products such as $a_1.\vec{i}$, we find nine integrals as follows.
\begin{align*}
\text{Coefficient of }a_1.\vec{i} =&\int_{\theta=0}^{\pi}\int_{\phi=0}^{2\pi}(\sin\theta)^3(\cos\phi)^2 d\theta d\phi\\
=&\int_{\theta=0}^{\pi}(\sin\theta)^3d\theta\int_{\phi=0}^{2\pi}(\cos\phi)^2  d\phi\\
=&\pi \int_{0}^{\pi}(\sin\theta)^3d\theta\text{, from Lemma \ref{lemma:i1}}\\
=&\pi \int_{0}^{\pi} [1-(\cos \theta)^2] \sin\theta d\theta  \\
=& \pi [-\cos\theta+\frac{1}{3}(\cos\theta)^3]\Bigg|_0^{\pi} \\
=& \pi [1-\frac{1}{3}] \\
=& \frac{2\pi}{3}\numberthis \label{eq:a1-i}
\end{align*}

\begin{align*}
\text{Coefficient of }a_2.\vec{i} =&\int_{\theta=0}^{\pi}\int_{\phi=0}^{2\pi}(\sin\theta)^3\sin\phi\cos\phi\ d\theta d\phi\\=&\int_{\theta=0}^{\pi}(\sin\theta)^3d\theta\int_{\phi=0}^{2\pi}\sin\phi\cos\phi\  d\phi\\
=&\int_{\theta=0}^{\pi}(\sin\theta)^3d\theta . \frac{1}{2} (\sin \phi)^2 \Big|_0^{2\pi} \\
=&\int_{\theta=0}^{\pi}(\sin\theta)^3d\theta\times 0\\
=&0 \numberthis \label{eq:a2-i}
\end{align*}

\begin{align*}
\text{Coefficient of }a_3.\vec{i} =&\int_{\theta=0}^{\pi}\int_{\phi=0}^{2\pi}\cos\theta(\sin\theta)^2\cos\phi d\theta d\phi\\=&\int_{\theta=0}^{\pi}\cos\theta(\sin\theta)^2d\theta\int_{\phi=0}^{2\pi}\cos\phi  d\phi\\
=&\int_{\theta=0}^{\pi}\cos\theta(\sin\theta)^2d\theta . \sin\phi \Big|_0^{2\pi}\\
=&\int_{\theta=0}^{\pi}\cos\theta(\sin\theta)^2d\theta \times 0\\
=&0 \numberthis \label{eq:a3-i}
\end{align*}

\begin{align*}
\text{Coefficient of }a_1.\vec{j} =&\int_{\theta=0}^{\pi}\int_{\phi=0}^{2\pi}(\sin\theta)^3\cos\phi\sin\phi d\theta d\phi\\=&\int_{\theta=0}^{\pi}(\sin\theta)^3d\theta\int_{\phi=0}^{2\pi}\cos\phi\sin\phi  d\phi\\
=&0 \text{, by a similar argument to (\ref{eq:a2-i})}\numberthis \label{eq:a1-j}
\end{align*}

\begin{align*}
\text{Coefficient of }a_2.\vec{j} =&\int_{\theta=0}^{\pi}\int_{\phi=0}^{2\pi}(\sin\theta)^3(\sin\phi)^2 d\theta d\phi\\=&\int_{\theta=0}^{\pi}(\sin\theta)^3 d\theta \int_{\phi=0}^{2\pi}(\sin\phi)^2  d\phi\\
=& \frac{2\pi}{3} \text{, by a similar argument to (\ref{eq:a1-i})} \numberthis \label{eq:a2-j}
\end{align*}

\begin{align*}
\text{Coefficient of }a_3.\vec{j} =&\int_{\theta=0}^{\pi}\int_{\phi=0}^{2\pi}\cos\theta(\sin\theta)^2\sin\phi  d\theta d\phi\\=&\int_{\theta=0}^{\pi}\cos\theta(\sin\theta)^2d\theta\int_{\phi=0}^{2\pi}\sin\phi   d\phi\\=&0 \text{, by a similar argument to (\ref{eq:a3-i})} \numberthis \label{eq:a3-j}
\end{align*}

\begin{align*}
\text{Coefficient of }a_1.\vec{k} =&\int_{\theta=0}^{\pi}\int_{\phi=0}^{2\pi}(\sin\theta)^2\cos\phi\cos\theta d\theta d\phi\\=&
\int_{\theta=0}^{\pi}  (\sin\theta)^2 cos\theta d\theta \int_{\phi=0}^{2\pi}\cos\phi\ d\phi\\
=&0 \text{, by a similar argument to (\ref{eq:a3-i})} \numberthis \label{eq:a1-k}
\end{align*}

\begin{align*}
\text{Coefficient of }a_2.\vec{k} =&\int_{\theta=0}^{\pi}\int_{\phi=0}^{2\pi}(\sin\theta)^2\sin\phi\cos\theta d\theta d\phi\\=&
\int_{\theta=0}^{\pi} (\sin\theta)^2 \cos\theta d\theta \int_{\phi=0}^{2\pi}\sin\phi d\phi\\=&0 \text{, by a similar argument to (\ref{eq:a2-i})} \numberthis \label{eq:a2-k}
\end{align*}

\begin{align*}
\text{Coefficient of }a_3.\vec{k} =&\int_{\theta=0}^{\pi} \int_{\phi=0}^{2\pi} (\cos\theta)^2\sin\theta d\theta   d\phi\\ &= \int_{\theta=0}^{\pi} (\cos\theta)^2\sin\theta d\theta  \int_{\phi=0}^{2\pi} d\phi \\ &= 2 \pi \int_{\theta=0}^{\pi} (\cos\theta)^2\sin\theta d\theta \\
=& 2\pi[-\frac{1}{3}(\cos\theta)^3] \Bigg|_0^{\pi} \\
=& \frac{2\pi}{3}\numberthis \label{eq:a3-k}
\end{align*}

Substituting (\ref{eq:a1-i}--\ref{eq:a3-k})  in (\ref{eq:big-int})
\begin{align*}
\left\langle \big(\vec{r}.\vec{a}\big)\vec{r} \right\rangle=&\frac{ R^2}{4\pi}\frac{2\pi}{3}\big[a_1.\vec{i}+a_2.\vec{j}+a_3.\vec{k}\big]\\
=&\frac{R^2}{6}\vec{a} \numberthis \label{eq:final}
\end{align*}
Hence $\mathfrak{R}$ in (\ref{eq:linear}) is scalar multiplication, rather than a more general linear transformation.

\begin{thebibliography}{9}
	\raggedright
	\bibitem{wiki-polar}
	Wikipedia contributors. Spherical coordinate system. Wikipedia, The Free Encyclopedia. August 19, 2017, 21:40 UTC. Available at: 
	\url{https://en.wikipedia.org/w/index.php?title=Spherical_coordinate_system&oldid=796300234}.
	Accessed
	October 13, 2017
\end{thebibliography}

\end{document}
