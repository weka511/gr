\documentclass[]{article}
\usepackage{esvect,mathtools,amssymb,amsthm,url,empheq}
\usepackage[nottoc,numbib]{tocbibind}
\newcommand\numberthis{\addtocounter{equation}{1}\tag{\theequation}}

%opening
\title{Introduction into General Relativity\\Assignment 10.2\\Gravitational Radiation}
\author{Simon Crase}

\begin{document}

\maketitle
\tableofcontents

\begin{abstract}
Find the change in time of the radius of a double--star system due to the gravitational radiation. Initial distance between the stars is $2R$ and their masses are $m_1$ and $m_2$.

\end{abstract}

\section{Assumptions}
I have made two simplifying assumptions:
\begin{enumerate}
	\item \label{assumption:1} The two stars begin in a circular orbit (the problem speaks of a radius);
	\item \label{assumption:2} The orbit remains circular (in reality the two stars spiral in, but we assume that the orbit can be treated instantaneously as a circle). 
\end{enumerate}

\section{The Quadrupole Tensor}
For two classical stars orbiting each other in circular orbits with masses $m_1$ and $m_2$, at a distance of $2R$ and angular velocity $\omega$, it is well known (e.g. \cite{wiki:kepler}) that:
\begin{align*}
8R^3\omega^2=\kappa(m_1+m_2) \numberthis \label{eq:kepler}
\end{align*}
Without loss of generality, we can choose Cartesian coordinates so that the motion takes place in the 1-2 plane, so there is no motion along the 3-axis. If we represents the two stars by vectors $\vec{y_{(1)}}$ and $\vec{y_{(2)}}$, we have:
\begin{empheq}[left=\empheqlbrace]{align*}
\vec{y_{(1)}}=&\bigg(\frac{2 m_2 R}{m_1+m_2}\, \cos (\omega t),\frac{2 m_2 R}{m_1+m_2}\, \sin (\omega t),0\bigg) \numberthis \label{eq:y}\\
\vec{y_{(2)}}=&\bigg(- \frac{2 m_1 R}{m_1+m_2}\, \cos (\omega t),- \frac{2 m_1 R}{m_1+m_2}\, \sin (\omega t),0\bigg) 
\end{empheq}
We know from \cite[Chapter X, Equation (250)]{Akhmedov2017} that:
\begin{align*}
Q_{ij}=& \int d^3 \vec{y}\,\rho(t-|\vec{x}|,\vec{y})\,(3y_iy_j-\delta_{ij}|\vec{(y)^2}|)\text{, where $\rho$ represents density}\\
\rho =& m_1 \delta(\vec{y}-\vec{y_{(1)}}) +  m_2 \delta(\vec{y}-\vec{y_{(2)}}) \text{, whence:}\\
Q_{ij}=&m_1\,(3y_{(1)i}y_{(1)j}-\delta_{ij}|\vec{(y_{1})^2}|)+m_2\,(3y_{(2)i}y_{(2)j}-\delta_{ij}|\vec{(y_{2})^2}|)\text{. Now }\\
m_1 |\vec{(y_{1})^2}| + m_2 |\vec{(y_{2})^2}|=&m_1 r_{(1)}^2 + m_2r_{(2)}^2\\
=&\frac{4 m_1 m_2^2R^2+4 m_2 m_1^2 R^2}{(m_1+m_2)^2}\\
=&\frac{4 m_1 m_2}{m_1 + m_2}R^2\numberthis \label{eq:delta}
\end{align*}
\begin{empheq}[left=\empheqlbrace]{align*}
	Q_{11}=&\,\bigg[\frac{3(m_1 4 m_2^2 R^2)}{(m_1+m_2)^2}+\frac{3(m_2 4 m_1^2 R^2)}{(m_1+m_2)^2}\bigg]\cos^2 (\omega t)-\frac{4 m_1 m_2}{m_1 + m_2}R^2 \numberthis \label{eq:Qs}\\
	=& \frac{12 m_1 m_2 R^2}{m_1 + m_2}\cos^2 (\omega t)-\frac{4 m_1 m_2}{m_1 + m_2}R^2\\
	Q_{12}=&\,\frac{12 m_1 m_2 R^2}{m_1 + m_2}\cos (\omega t) \sin(\omega t)\\
	Q_{13}=&\,0\\
	Q_{21}=&\,\frac{12 m_1 m_2 R^2}{m_1 + m_2}\cos (\omega t) \sin(\omega t)\\
	Q_{22}=&\,\frac{12 m_1 m_2 R^2}{m_1 + m_2}\sin^2 (\omega t)-\frac{4 m_1 m_2}{m_1 + m_2}R^2\\
	Q_{23}=&\,0\\
	Q_{31}=&\,0\\
	Q_{32}=&\,0\\
	Q_{33}=&\,-\frac{4 m_1 m_2}{m_1 + m_2}R^2\text{, from ref \eqref{eq:delta}}	
\end{empheq}

\section{Radiation intensity}
From \cite[X,(258)]{Akhmedov2017},
\begin{equation}
I\approxeq\frac{\kappa}{45}\dddot{Q}_{ij}\dddot{Q}_{ij} \label{eq:I}
\end{equation}
There are only four terms in \eqref{eq:Qs} that can have non-zero derivatives. Using well known trigonometric identities to simplify differentiation we have:
\begin{empheq}[left=\empheqlbrace]{align*}
Q_{11}=& \frac{6 m_1 m_2 R^2}{m_1 + m_2}\big(1+\cos (2\omega t)\big)-\frac{4 m_1 m_2}{m_1 + m_2}R^2\\
Q_{12}=&\,\frac{6 m_1 m_2 R^2}{m_1 + m_2} \sin(2\omega t)\\
Q_{21}=&\,\frac{6 m_1 m_2 R^2}{m_1 + m_2} \sin(2\omega t)\\
Q_{22}=&\,\frac{6 m_1 m_2 R^2}{m_1 + m_2}\big(1-\cos (2\omega t)\big)-\frac{4 m_1 m_2}{m_1 + m_2}R^2
\end{empheq}

\begin{empheq}[left=\empheqlbrace]{align*}
\dddot{Q}_{11}=& \frac{48 \omega^3 m_1 m_2 R^2}{m_1 + m_2}\sin (2\omega t)\\
\dddot{Q}_{12}=&-\,\frac{48 \omega^3 m_1 m_2 R^2}{m_1 + m_2} \cos(2\omega t)\\
\dddot{Q}_{21}=&-\,\frac{48 \omega^3 m_1 m_2 R^2}{m_1 + m_2} \cos(2\omega t)\\
\dddot{Q}_{22}=&-\,\frac{48 \omega^3 m_1 m_2 R^2}{m_1 + m_2}\sin (2\omega t)
\end{empheq}

So \eqref{eq:I} becomes:
\begin{align*}
- \dot{E}=&I\text{, where $E$ is the total energy of the system}\\
\approxeq&\frac{\kappa}{45}\Big(\,\frac{48 \omega^3 m_1 m_2 R^2}{m_1 + m_2}\Big)^2\Big[\sin^2 (2\omega t)+\cos^2 (2\omega t)+\cos^2 (2\omega t)\sin^2 (2\omega t)\Big]\\
\approxeq& 2 \frac{\kappa}{45}\Big(\,\frac{48 \omega^3 m_1 m_2 R^2}{m_1 + m_2}\Big)^2  \\
\approxeq&\frac{512 \kappa \, \omega^6 m_1^2 m_2^2 R^4}{5\,(m_1 + m_2)^2}  \numberthis  \label{eq:I-derived}
\end{align*}


\section{Velocity of the distance decrease}
Now the energy of two stars in the orbit is (e.g. \cite{goldstein})
\begin{align*}
E=& T + V\text{, where $T$ is the kinetic energy and $V$ the potential energy} \numberthis \label{eq:E}\\
T=&\frac{1}{2}\Big(m_1r_1^2+m_2r_2^2\Big) \omega^2\\
=& \frac{1}{2}\Big(\frac{m_1 4m_2^2R^2}{(m_1+m_2)^2}+\frac{m_2 4m_1^2R^2}{(m_1+m_2)^2}\Big)\\
=&\frac{2 m_1 m_2 R^2 \omega^2}{m_1 + m_2}\\
=& \frac{2 m_1 m_2}{m_1 + m_2} \frac{\kappa (m_1 + m_2)}{8 R}\text{, from \eqref{eq:kepler}}\\
=& \frac{\kappa m_1 m_2}{4R} \numberthis \label{eq:T}\\
V=& - \frac{\kappa m_1 m_2}{2 R} \text{(e.g. \cite{goldstein})} \numberthis \label{eq:V}\\
E=&-\frac{\kappa m_1 m_2}{4R}\text{, using \eqref{eq:T}, \eqref{eq:V}, and \eqref{eq:E}. Differentiating and using Assumption \ref{assumption:2}:} \\
\dot{E} =& \frac{\kappa m_1 m_2}{4R^2}\dot{R}\text{, whence}\\
\dot{R}=& \frac{4 R^2}{\kappa m_1 m_2} \dot{E} \text{ and }\\
\approxeq& - \frac{4 R^2}{\kappa m_1 m_2} \frac{512\kappa\, \omega^6 m_1^2 m_2^2 R^4}{5\,(m_1 + m_2)^2} \text{, from \eqref{eq:I-derived}}\\
\approxeq& - \frac{2048 \omega^6 m_1 m_2 R^6}{5\,(m_1 + m_2)^2}
\end{align*}
\section{Time dependence of the distance}

So, in time $\Delta t$ we expect $2R$ to reduce to $2R-\frac{2048 \omega^6 m_1 m_2 R^6}{5\,(m_1 + m_2)^2} \Delta t$.

\begin{thebibliography}{9}
	\bibitem{Akhmedov2017}
	Emil T. Akhmedov,
	\emph{Lectures on General Theory of Relativity},
 	arXiv:1601.04996,
 	\url{https://arxiv.org/abs/1601.04996}
 	\bibitem{wiki:kepler}
 	Wikipedia contributors.
 	\emph{Kepler's laws of planetary motion. Wikipedia, The Free Encyclopedia.}  October 10, 2017, 01:47 UTC. Available at: 
 	\url{https://en.wikipedia.org/w/index.php?title=Kepler%27s_laws_of_planetary_motion&oldid=804602772.} Accessed
 	October 17, 2017
 	\bibitem{goldstein}
 	Herbert Goldstein,
 	\emph{Classical Mechanics},
 	 Addison-Wesley. 1951.
 	 ASIN B000OL8LOM
\end{thebibliography}
\end{document}
