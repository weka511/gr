\documentclass[]{article}
\usepackage{amsmath,url,amssymb}
\newcommand\numberthis{\addtocounter{equation}{1}\tag{\theequation}}
%opening
\title{Introduction into General Relativity\\Assignment 12.1\\Cosmological solutions with $\Lambda\ne 0$}
\author{Simon Crase}

\begin{document}

\maketitle

\begin{abstract}
What is the result of the Wick rotation of the Rindler metric?.
\begin{itemize}
	\item  Appropriate equivalent signature of the Rindler's metric is chosen
	\item  Correct metric after the Wick rotation is obtained
	\item  Correct interpretation of the obtained metric is given
\end{itemize}
\end{abstract}

\section{Na\"ive Approach}
From the Lecture Notes, \cite[I, (9) \& (10)]{Akhmedov2017}, the Rindler metric is obtained from the Minkowski metric, by writing:
\begin{align*}
t=&\, \rho \sinh \tau \numberthis\label{eq:t_tau}\\
x=&\, \rho \cosh \tau \text{, whence} \numberthis\label{eq:r_rho}\\
ds^2 =&\, dt^2-dx^2-dy^2-dz^2\text{ becomes}\\
=&\, \rho^2 d\tau^2 - d\rho ^2 -dy^2-dz^2\numberthis\label{eq:rindler}
\end{align*}

We apply Wick Rotation by introducing a new imaginary time, $\mathfrak{T}$:
\begin{align*}
\mathfrak{T}=&i\tau\text{, whence \eqref{eq:rindler} becomes}\\
ds^2 =& -\rho^2 d\mathfrak{T}^2 - d\rho^2-dy^2-dz^2\text{, a metric with signature (-1,-1,-1,-1)}
\end{align*}
The problem is that $ds^2<0$, leading to an imaginary $ds$!

\section{Making $ds^2>0$}
We follow \cite{Akhmedov2017} and reverse the sign of the metric, $ds^2=-\eta_{\mu\nu}x^{\mu}x^{\nu}$. Then, applying Wick rotation:
\begin{equation*}
ds^2 = \rho^2 d\mathfrak{T}^2 + d\rho^2+dy^2+dz^2\text{, a metric with signature (+1,+1,+1,+1)}.
\end{equation*}
\begin{thebibliography}{9}\label{section:biblio}
	\raggedright
	\bibitem{Akhmedov2017}
	Emil T. Akhmedov,
	\emph{Lectures on General Theory of Relativity},
	arXiv:1601.04996,
	\url{https://arxiv.org/abs/1601.04996}
\end{thebibliography}

\end{document}
