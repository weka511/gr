\documentclass[]{article}
\usepackage{amsmath,amsthm,esvect,amssymb,url}
\usepackage[nottoc,numbib]{tocbibind}
\newtheorem{lemma}{Lemma}
\newcommand\numberthis{\addtocounter{equation}{1}\tag{\theequation}}
%opening
\title{Introduction into General Relativity\\Assignment 11\\Friedman-Robertson-Walker cosmology}
\author{Simon Crase}

\begin{document}

\maketitle

\begin{abstract}
Calculate the Ricci tensor for the following Euclidean metrics:
\begin{align*}
dl^2\equiv&\gamma_{ij}dx^idx^j=\frac{dr^2}{1-r^2}+r^2\,d\Omega^2\\
dl^2\equiv&d\chi^2+\sin^2 \chi\,d\Omega^2\\
dl^2\equiv&\frac{dr^2}{1+r^2}+r^2\,d\Omega^2\\
dl^2\equiv&d\chi^2+\sinh^2 \chi\,d\Omega^2\text{, where}\\
d\Omega^2\triangleq & d\theta^2 + r^2 d\phi^2
\end{align*}
Show that they solve 3d Einstein equations in Euclidean signature $R_{ij}(\gamma)=2k\gamma_{ij}$ for $k=\pm 1$ and $R_{ij}(\gamma)$ is the Ricci tensor for the metric $\gamma$.
\end{abstract}

\tableofcontents

\section{Metric: $dl^2\equiv\frac{dr^2}{1-r^2}+r^2\,d\Omega^2$}
We will use the following well known result.
\begin{lemma}\label{lemma:geodesic}
	The following to methods of defining a geodesic are equivalent.
	\begin{align*}
	\ddot{z}_i + \Gamma^i_{jk}\dot{z}_j \dot{z}_k =& 0 \\
	\frac{\partial}{\partial \tau}\Big(\frac{\partial L}{\partial \dot{z}^i}\Big)-\frac{\partial L}{\partial z^i}=&0.
	\end{align*}
	where the metric is defined by $ds^2\equiv L(\vec{z}) d\tau^2$ 
\end{lemma}
We can save some effort if we combine this metric with that of Section \ref{section:metric3} by writing:
\begin{align*}
L=&(1+\alpha r^2)^{-1}\dot{r}^2 + r^2(\dot\theta^2 + \sin^2 \theta \dot \phi^2)\text{, where $\alpha=\pm 1$} \numberthis \label{eq:lagrangian1}\\
\frac{\partial}{\partial \tau}\Big(\frac{\partial L}{\partial \dot{r}}\Big)=&\frac{\partial}{\partial \tau}\Big(2(1+\alpha r^2)^{-1}\dot r\Big)\\
=& \Big(2(1+\alpha r^2)^{-1}\ddot r\Big)-4 \alpha r (1+\alpha r^2)^{-2}\dot{r}^2\\
\frac{\partial L}{\partial r}=&2r\big(\dot{\theta}^2 + \sin^2 \theta \dot \phi ^2\big)\\
0=&\frac{\partial}{\partial \tau}\Big(\frac{\partial L}{\partial \dot{r}}\Big)-\frac{\partial L}{\partial r}\\
\iff&\\
0=&	\Big(2(1+\alpha r^2)^{-1}\ddot r\Big)-4 \alpha r (1+\alpha r^2)^{-2}\dot{r}^2-2r\big(\dot{\theta}^2 + \sin^2 \theta \dot \phi ^2\big)\\
\iff&\\
0=&\ddot r-2 \alpha r (1+\alpha r^2)^{-1}\dot{r}^2-r(1+\alpha r^2)\big(\dot{\theta}^2 + \sin^2 \theta \dot \phi ^2\big)
\end{align*}
We can use Lemma \ref{lemma:geodesic} to read off the $\Gamma^r_{ij}$
\begin{align*}
\Gamma^r_{rr}=&-2 \alpha r (1+\alpha r^2)^{-1}\\
\Gamma^r_{\theta\theta}=&r(1+\alpha r^2)\\
\Gamma^r_{\phi\phi}=&r(1+\alpha r^2)\sin^2 \theta\\
\Gamma^r_{ij}=& 0\text{, for $i\ne j$}
\end{align*}
Similarly
\begin{align*}
0=&\frac{\partial}{\partial \tau}\Big(\frac{\partial L}{\partial \dot{\theta}}\Big)-\frac{\partial L}{\partial \theta}\\
\iff&\\
0=&\frac{\partial}{\partial \tau}\Big(2r^2 \dot {\theta} \Big) -2 r^2 \sin \theta\cos \theta \dot{\phi}^2\\
=&4 r \dot r  \dot{\theta}  + 2 r^2 \ddot \theta -2 r^2 \sin \theta\cos \theta \dot{\phi}^2\\
=& 2r^2\Big(\ddot{\theta}+\frac{2}{r}\dot{\theta}\dot{r} -\sin \theta\cos \theta \dot{\phi}^2\Big)\text{, whence, from Lemma \ref{lemma:geodesic}}\\
\Gamma^{\theta}_{r\theta}=&-\frac{1}{r}\\
\Gamma^{\theta}_{\theta r}=&-\frac{1}{r}\\
\Gamma^{\theta}_{\phi\phi}=&\sin \theta \cos \theta\\
\Gamma^{\theta}_{ij}=&0 \text{ otherwise.}
\end{align*}

and
\begin{align*}
0=&\,\frac{\partial}{\partial \tau}\Big(\frac{\partial L}{\partial \dot{\phi}}\Big)-\frac{\partial L}{\partial \phi}\\
\iff&\\
0=&\frac{\partial}{\partial \tau}\Big( 2 r^2 sin^2\theta \dot{\phi}\Big) - 0\\
=&4 r \sin^2 \theta \dot {r} \dot{\phi}+ 4 r^2 sin\theta cos\theta \dot{\theta} \dot{\phi}+2 r^2 sin^2\theta \ddot{\phi}\\
=& 2 r^2 sin^2\theta \Big(\ddot{\phi} + \frac{2}{r}\dot{r}\dot{\phi} + 2 \cot\theta\dot{\theta}\dot{\phi}\Big)\text{, whence, from Lemma \ref{lemma:geodesic}}\\
\Gamma^{\phi}_{r\phi}=&\frac{1}{r}\\
\Gamma^{\phi}_{\phi r}=&\frac{1}{r}\\
\Gamma^{\phi}_{\theta\phi}=&\cot\theta\\
\Gamma^{\phi}_{\phi\theta}=&\cot\theta\\
\Gamma^{\phi}_{ij}=&0 \text{, otherwise.}
\end{align*}

Now the Ricci Tensor is given in \cite[II, (58) and (56)]{Akhmedov2017} to be:
\begin{align*}
R_{jl}\equiv&\, R^i_{\,jil}\text{, where}\\
R^i_{\,jkl}\equiv&\,\partial_k\Gamma^i_{jl}-\partial_l\Gamma^i_{jk}+\Gamma^i_{mk} \Gamma^m_{jl} - \Gamma^i_{ml} \Gamma^m_{jk}\text{, whence}\\
R_{jl}=&\,\partial_i\Gamma^i_{jl}-\partial_l\Gamma^i_{ji}+\Gamma^k_{mk} \Gamma^m_{jl} - \Gamma^k_{ml} \Gamma^m_{jk}\\
=&\,\partial_i\Gamma^i_{jl}-\frac{1}{2|g|}\big(\partial_l \, |g|\big)+\frac{1}{2|g|}\big(\partial_m \, |g|\big) \Gamma^m_{jl} - \Gamma^k_{ml} \Gamma^m_{jk}\text{since $\Gamma^{\alpha}_{\rho\alpha}=\frac{1}{2|g|}\big(\partial_\rho \, |g|\big)$ \cite{abs1965}.}
\end{align*}
From \eqref{eq:lagrangian1}
\begin{align*}
|g|=\frac{r^4\sin^2 \theta}{1+\alpha r^2}
\end{align*}




So, setting $\alpha=-1$
\section{Metric: $dl^2\equiv d\chi^2+\sin^2 \chi\,d\Omega^2$}
\section{Metric: $dl^2\equiv\frac{dr^2}{1+r^2}+r^2\,d\Omega^2$}  \label{section:metric3}
So, setting $\alpha=+1$
\section{Metric: $dl^2\equiv d\chi^2+\sinh^2 \chi\,d\Omega^2$}

\begin{thebibliography}{9}\label{section:biblio}
	\raggedright
	\bibitem{Akhmedov2017}
	Emil T. Akhmedov,
	\emph{Lectures on General Theory of Relativity},
	arXiv:1601.04996,
	\url{https://arxiv.org/abs/1601.04996}
	\bibitem{abs1965}
	Ronald Adler, Maurice Bazin, \& Menahem Schiffer,
	\emph{Introduction to General Relativity},
	McGraw-Hill Book Company, New York,
	1965.
\end{thebibliography}

\end{document}
