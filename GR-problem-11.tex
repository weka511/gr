\documentclass[]{article}
\usepackage{amsmath,amsthm,esvect,amssymb,url}
\usepackage[nottoc,numbib]{tocbibind}
\newtheorem{lemma}{Lemma}
\newcommand\numberthis{\addtocounter{equation}{1}\tag{\theequation}}
%opening
\title{Introduction into General Relativity\\Assignment 11\\Friedman-Robertson-Walker cosmology}
\author{Simon Crase}

\begin{document}

\maketitle

\begin{abstract}
Calculate the Ricci tensor for the following Euclidean metrics:
\begin{align*}
dl^2\equiv&\gamma_{ij}dx^idx^j=\frac{dr^2}{1-r^2}+r^2\,d\Omega^2\\
dl^2\equiv&d\chi^2+\sin^2 \chi\,d\Omega^2\\
dl^2\equiv&\frac{dr^2}{1+r^2}+r^2\,d\Omega^2\\
dl^2\equiv&d\chi^2+\sinh^2 \chi\,d\Omega^2\text{, where}\\
d\Omega^2\triangleq & d\theta^2 + r^2 d\phi^2
\end{align*}
Show that they solve 3d Einstein equations in Euclidean signature $R_{ij}(\gamma)=2k\gamma_{ij}$ for $k=\pm 1$ and $R_{ij}(\gamma)$ is the Ricci tensor for the metric $\gamma$.
\end{abstract}

\tableofcontents

\section{Metric: $dl^2\equiv\frac{dr^2}{1-r^2}+r^2\,d\Omega^2$} \label{section:metric1}
We will use the following well known result.
\begin{lemma}\label{lemma:geodesic}
	The following to methods of defining a geodesic are equivalent.
	\begin{align*}
	\ddot{z}_i + \Gamma^i_{jk}\dot{z}_j \dot{z}_k =& 0 \\
	\frac{d}{d \tau}\Big(\frac{\partial L}{\partial \dot{z}^i}\Big)-\frac{\partial L}{\partial z^i}=&0.
	\end{align*}
	where the metric is defined by $ds^2\equiv L(\vec{z}) d\tau^2$ 
\end{lemma}
We can save some effort if we combine this metric with that of Section \ref{section:metric3} by writing:
\begin{align*}
L=&(1+\alpha r^2)^{-1}\dot{r}^2 + r^2(\dot\theta^2 + \sin^2 \theta \dot \phi^2)\text{, where $\alpha=\pm 1$} \numberthis \label{eq:lagrangian1}\\
\frac{d}{d \tau}\Big(\frac{\partial L}{\partial \dot{r}}\Big)=&\frac{d}{d \tau}\Big(2(1+\alpha r^2)^{-1}\dot r\Big)\\
=& \Big(2(1+\alpha r^2)^{-1}\ddot r\Big)-4 \alpha r (1+\alpha r^2)^{-2}\dot{r}^2\\
\frac{\partial L}{\partial r}=&2r\big(\dot{\theta}^2 + \sin^2 \theta \dot \phi ^2\big)\\
0=&\frac{d}{d \tau}\Big(\frac{\partial L}{\partial \dot{r}}\Big)-\frac{\partial L}{\partial r}\\
\iff&\\
0=&	\Big(2(1+\alpha r^2)^{-1}\ddot r\Big)-4 \alpha r (1+\alpha r^2)^{-2}\dot{r}^2-2r\big(\dot{\theta}^2 + \sin^2 \theta \dot \phi ^2\big)\\
\iff&\\
0=&\ddot r-2 \alpha r (1+\alpha r^2)^{-1}\dot{r}^2-r(1+\alpha r^2)\big(\dot{\theta}^2 + \sin^2 \theta \dot \phi ^2\big)
\end{align*}
We can use Lemma \ref{lemma:geodesic} to read off the $\Gamma^r_{ij}$
\begin{align*}
\Gamma^r_{rr}=&-2 \alpha r (1+\alpha r^2)^{-1}\\
\Gamma^r_{\theta\theta}=&r(1+\alpha r^2)\\
\Gamma^r_{\phi\phi}=&r(1+\alpha r^2)\sin^2 \theta\\
\Gamma^r_{ij}=& 0\text{, for $i\ne j$}
\end{align*}
Similarly
\begin{align*}
0=&\frac{d}{d \tau}\Big(\frac{\partial L}{\partial \dot{\theta}}\Big)-\frac{\partial L}{\partial \theta}\\
\iff&\\
0=&\frac{d}{d \tau}\Big(2r^2 \dot {\theta} \Big) -2 r^2 \sin \theta\cos \theta \dot{\phi}^2\\
=&4 r \dot r  \dot{\theta}  + 2 r^2 \ddot \theta -2 r^2 \sin \theta\cos \theta \dot{\phi}^2\\
=& 2r^2\Big(\ddot{\theta}+\frac{2}{r}\dot{\theta}\dot{r} -\sin \theta\cos \theta \dot{\phi}^2\Big)\text{, whence, from Lemma \ref{lemma:geodesic}}\\
\Gamma^{\theta}_{r\theta}=&-\frac{1}{r}\\
\Gamma^{\theta}_{\theta r}=&-\frac{1}{r}\\
\Gamma^{\theta}_{\phi\phi}=&\sin \theta \cos \theta\\
\Gamma^{\theta}_{ij}=&0 \text{ otherwise.}
\end{align*}

and
\begin{align*}
0=&\,\frac{d}{d \tau}\Big(\frac{\partial L}{\partial \dot{\phi}}\Big)-\frac{\partial L}{\partial \phi}\\
\iff&\\
0=&\frac{d}{d \tau}\Big( 2 r^2 sin^2\theta \dot{\phi}\Big) - 0\\
=&4 r \sin^2 \theta \dot {r} \dot{\phi}+ 4 r^2 sin\theta cos\theta \dot{\theta} \dot{\phi}+2 r^2 sin^2\theta \ddot{\phi}\\
=& 2 r^2 sin^2\theta \Big(\ddot{\phi} + \frac{2}{r}\dot{r}\dot{\phi} + 2 \cot\theta\dot{\theta}\dot{\phi}\Big)\text{, whence, from Lemma \ref{lemma:geodesic}}\\
\Gamma^{\phi}_{r\phi}=&\frac{1}{r}\\
\Gamma^{\phi}_{\phi r}=&\frac{1}{r}\\
\Gamma^{\phi}_{\theta\phi}=&\cot\theta\\
\Gamma^{\phi}_{\phi\theta}=&\cot\theta\\
\Gamma^{\phi}_{ij}=&0 \text{, otherwise.}
\end{align*}

Now the Ricci Tensor is given in \cite[II, (58) and (56)]{Akhmedov2017} to be:
\begin{align*}
R_{jl}\equiv&\, R^i_{\,jil}\text{, where}\\
R^i_{\,jkl}\equiv&\,\partial_k\Gamma^i_{jl}-\partial_l\Gamma^i_{jk}+\Gamma^i_{mk} \Gamma^m_{jl} - \Gamma^i_{ml} \Gamma^m_{jk}\text{, whence}\\
R_{jl}=&\,\partial_i\Gamma^i_{jl}-\partial_l\Gamma^i_{ji}+\Gamma^k_{mk} \Gamma^m_{jl} - \Gamma^k_{ml} \Gamma^m_{jk}\\
=&\,\partial_i\Gamma^i_{jl}-\frac{1}{2|g|}\big(\partial_l \, |g|\big)+\frac{1}{2|g|}\big(\partial_m \, |g|\big) \Gamma^m_{jl} - \Gamma^k_{ml} \Gamma^m_{jk}\text{since $\Gamma^{\alpha}_{\rho\alpha}=\frac{1}{2|g|}\big(\partial_\rho \, |g|\big)$ \cite{abs1965}.}
\end{align*}
From \eqref{eq:lagrangian1}
\begin{align*}
|g|=\frac{r^4\sin^2 \theta}{1+\alpha r^2}
\end{align*}




So, setting $\alpha=-1$
\section{Metric: $dl^2\equiv d\chi^2+\sin^2 \chi\,d\Omega^2$} \label{section:metric2}
We will use the well known formula $\sinh(\chi)=-i \sin(i\chi)$, \cite{wiki:sinh}, to reduce the metrics of Sections \ref{section:metric2} and \ref{section:metric4} to a common form.
\begin{align*}
L=&\dot{\chi}^2 + \alpha \sin^2(\beta \chi)(\dot\theta^2 + \sin^2 \theta \dot \phi^2)\text{, where $\alpha=1,\beta=1$ or $\alpha=-1,\beta=i$} \numberthis \label{eq:lagrangian2}\\
0=&\frac{d}{d \tau}\Big(\frac{\partial L}{\partial \dot{\chi}}\Big)-\frac{\partial L}{\partial \chi}\\
=&\ddot{\chi}-2\alpha\beta \cos{\beta\chi}\sin{\beta\chi}(\dot\theta^2 + \sin^2 \theta \dot \phi^2)\text{, whence:}\\
\Gamma^{\chi}_{\theta\theta}=&2\alpha\beta \cos{\beta\chi}\sin{\beta\chi}\\
\Gamma^{\chi}_{\phi\phi}=&2\alpha\beta \cos{\beta\chi}\sin{\beta\chi}\sin^2 \theta\\
\Gamma^{\chi}_{jk}=&0
\end{align*}

\begin{align*}
0=&\frac{d}{d \tau}\Big(\frac{\partial L}{\partial \dot{\theta}}\Big)-\frac{\partial L}{\partial \theta}\\
=&\frac{d}{d \tau}\big(2\alpha\sin^2{(\beta\chi)}\dot{\theta}\big)-2\alpha\sin^2{(\beta\chi)}\sin{(\theta)}\cos{(\theta)}\dot{\phi}^2\\
=&2\alpha\big[\sin^2{(\beta\chi)}\ddot{\theta}+\beta\sin{(\beta\chi)}\cos{(\beta\chi)}\dot{\chi}\dot{\theta}-\sin^2{(\beta\chi)}\sin{(\theta)}\cos{(\theta)}\dot{\phi}^2\big]\\
=&2\alpha\sin^2{(\beta\chi)}\big[\ddot{\theta}+\cot{(\beta\chi)}\dot{\chi}\dot{\theta}-\sin{(\theta)}\cos{(\theta)}\dot{\phi}^2\big]\text{, whence}\\
\Gamma^{\theta}_{\chi\theta}=&\frac{1}{2}cot{(\beta\chi)}\\
\Gamma^{\theta}_{\theta\chi}=&\frac{1}{2}cot{(\beta\chi)}\\
\Gamma^{\theta}_{\phi\phi}=&-\sin{(\theta)}\cos{(\theta)}\\
\Gamma^{\theta}_{ij}=&0\text{ otherwise}
\end{align*}

\begin{align*}
0=&\frac{d}{d \tau}\Big(\frac{\partial L}{\partial \dot{\phi}}\Big)-\frac{\partial L}{\partial \phi}\\
=&2\frac{d}{d \tau}\big(\alpha \sin^2(\beta \chi) \sin^2 \theta \dot \phi\big)\\
=&2\big[\big(\alpha \sin^2(\beta \chi) \sin^2 \theta \ddot \phi\big) + \big(2 \alpha \beta \sin(\beta \chi) \cos(\beta \chi) \sin^2 \theta \dot{\chi}\dot \phi\big) + \big(2\alpha \sin^2(\beta \chi) \sin \theta \cos\theta \dot\theta \dot \phi\big)\big]\\
=&2\alpha \sin^2(\beta \chi) \sin^2 \theta\big[\ddot{\phi}+2 \beta \cot{(\beta\chi)} \dot{\theta}\dot{\phi}+2\cot{\theta\dot\theta \dot \phi}\big]\text{, whence:}\\
\Gamma^{\phi}_{\theta\chi}=&\beta\cot{(\beta\chi)}\\
\Gamma^{\phi}_{\chi\theta}=&\beta\cot{(\beta\chi)}\\
\Gamma^{\phi}_{\theta\phi}=&cot{(\theta)}\\
\Gamma^{\phi}_{\theta\phi}=&cot{(\theta)}\\
\Gamma^{\phi}_{ij}=&0\text{, otherwise}
\end{align*}
From \eqref{eq:lagrangian2}

\begin{align*}
|g|=\alpha^2sin^4{(\beta\chi)}sin^2{(\theta)}
\end{align*}

So setting $\alpha=1,\beta=1$ 
\section{Metric: $dl^2\equiv\frac{dr^2}{1+r^2}+r^2\,d\Omega^2$}  \label{section:metric3}
So, setting $\alpha=+1$
\section{Metric: $dl^2\equiv d\chi^2+\sinh^2 \chi\,d\Omega^2$} \label{section:metric4}

So, setting $\alpha=-1,\beta=i$
\begin{thebibliography}{9}\label{section:biblio}
	\raggedright
	\bibitem{Akhmedov2017}
	Emil T. Akhmedov,
	\emph{Lectures on General Theory of Relativity},
	arXiv:1601.04996,
	\url{https://arxiv.org/abs/1601.04996}
	\bibitem{abs1965}
	Ronald Adler, Maurice Bazin, \& Menahem Schiffer,
	\emph{Introduction to General Relativity},
	McGraw-Hill Book Company, New York,
	1965.
	\bibitem{wiki:sinh}
	Wikipedia contributors
	\emph{Hyperbolic function. Wikipedia, The Free Encyclopedia. September 25, 2017, 01:29 UTC.}
	 Available at: 
	 \url{https://en.wikipedia.org/w/index.php?title=Hyperbolic_function&oldid=802265878}. Accessed October 21, 2017
\end{thebibliography}

\end{document}
