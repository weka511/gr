\documentclass[]{article}
\usepackage{amsmath,amsthm,esvect,amssymb,url,empheq,tensor,cancel}
\usepackage[nottoc,numbib]{tocbibind}
\newtheorem{lemma}{Lemma}
\newcommand\numberthis{\addtocounter{equation}{1}\tag{\theequation}}

%opening
\title{Introduction into General Relativity\\Assignment 11\\Friedman-Robertson-Walker cosmology}
\author{Simon Crase}

\begin{document}

\maketitle

\begin{abstract}
Calculate the Ricci tensor for the following Euclidean metrics:
\begin{align*}
dl^2\equiv&\gamma_{ij}dx^idx^j=\frac{dr^2}{1-r^2}+r^2\,d\Omega^2\\
dl^2\equiv&d\chi^2+\sin^2 \chi\,d\Omega^2\\
dl^2\equiv&\frac{dr^2}{1+r^2}+r^2\,d\Omega^2\\
dl^2\equiv&d\chi^2+\sinh^2 \chi\,d\Omega^2
\end{align*}
Show that they solve 3d Einstein equations in Euclidean signature $R_{ij}(\gamma)=2k\gamma_{ij}$ for $k=\pm 1$ and $R_{ij}(\gamma)$ is the Ricci tensor for the metric $\gamma$.
\end{abstract}

\tableofcontents

\section{Preliminaries}

We will use the following well known result.
\begin{lemma}\label{lemma:geodesic}
	The following two methods of defining a geodesic are equivalent.
	\begin{align*}
	\ddot{z}_i + \Gamma^i_{jk}\dot{z}_j \dot{z}_k =& 0 \\
	\frac{d}{d \tau}\Big(\frac{\partial L}{\partial \dot{z}^i}\Big)-\frac{\partial L}{\partial z^i}=&0.
	\end{align*}
	where the metric is defined by $ds^2\equiv L(\vec{z}) d\tau^2$ 
\end{lemma}

\begin{lemma}\label{lemma:independent-components}
	In a 3 dimensional space, $R_{ijkl}$ has at most 6 independent components.
\end{lemma}
\begin{proof}
	In 3 dimensional space, $R_{ijkl}$ has 81 components, but we can save computational labour by exploiting symmetry properties.
	\begin{align}
	R_{ijkl}=&-R_{ijlk}\label{eq:as1}\\
	=&-R_{jikl}\label{eq:as2}\\
	=&\,R_{klij}\label{eq:s1}
	\end{align}
	
	The antisymmetry conditions, \eqref{eq:as1} and \eqref{eq:as2}, reduce the number of independent combinations of indices from 81 to 9, i.e. 3 combinations of $\{i,j\}$ and 3 of $\{k,l\}$, so the number of independent combinations of indices is bounded above by 9. The symmetry condition, \eqref{eq:s1}, reduces the upper bound further to 6.
\end{proof}

\section{Metric: $dl^2\equiv\frac{dr^2}{1-r^2}+r^2\,d\Omega^2$} \label{section:metric1}

\subsection{Christoffel Symbols}
We can save some effort if we combine this metric with that of Section \ref{section:metric3} by writing:
\begin{align*}
L=&(1+\alpha r^2)^{-1}\dot{r}^2 + r^2(\dot\theta^2 + \sin^2 \theta \dot \phi^2)\text{, where $\alpha=\pm 1$} \numberthis \label{eq:lagrangian1}\\
\frac{d}{d \tau}\Big(\frac{\partial L}{\partial \dot{r}}\Big)=&\frac{d}{d \tau}\Big(2(1+\alpha r^2)^{-1}\dot r\Big)\\
=& \Big(2(1+\alpha r^2)^{-1}\ddot r\Big)-4 \alpha r (1+\alpha r^2)^{-2}\dot{r}^2\\
\frac{\partial L}{\partial r}=&2r\big(\dot{\theta}^2 + \sin^2 \theta \dot \phi ^2\big)\\
0=&\frac{d}{d \tau}\Big(\frac{\partial L}{\partial \dot{r}}\Big)-\frac{\partial L}{\partial r}\\
\iff&\\
0=&	\Big(2(1+\alpha r^2)^{-1}\ddot r\Big)-4 \alpha r (1+\alpha r^2)^{-2}\dot{r}^2-2r\big(\dot{\theta}^2 + \sin^2 \theta \dot \phi ^2\big)\\
\iff&\\
0=&\ddot r-2 \alpha r (1+\alpha r^2)^{-1}\dot{r}^2-r(1+\alpha r^2)\big(\dot{\theta}^2 + \sin^2 \theta \dot \phi ^2\big)
\end{align*}
We can use Lemma \ref{lemma:geodesic} to read off the $\Gamma^r_{ij}$
\begin{empheq}[left=\empheqlbrace]{align*}\numberthis\label{eq:1-gamma-r}
\Gamma^r_{rr}=&-2 \alpha r (1+\alpha r^2)^{-1}\\
\Gamma^r_{\theta\theta}=&-r(1+\alpha r^2)\\
\Gamma^r_{\phi\phi}=&-r(1+\alpha r^2)\sin^2 \theta\\
\Gamma^r_{ij}=& 0\text{, for $i\ne j$}
\end{empheq}
Similarly
\begin{align*}
0=&\frac{d}{d \tau}\Big(\frac{\partial L}{\partial \dot{\theta}}\Big)-\frac{\partial L}{\partial \theta}\\
\iff&\\
0=&\frac{d}{d \tau}\Big(2r^2 \dot {\theta} \Big) -2 r^2 \sin \theta\cos \theta \dot{\phi}^2\\
=&4 r \dot r  \dot{\theta}  + 2 r^2 \ddot \theta -2 r^2 \sin \theta\cos \theta \dot{\phi}^2\\
=& 2r^2\Big(\ddot{\theta}+\frac{2}{r}\dot{\theta}\dot{r} -\sin \theta\cos \theta \dot{\phi}^2\Big)\text{, whence, from Lemma \ref{lemma:geodesic}}
\end{align*}

\begin{empheq}[left=\empheqlbrace]{align*}\numberthis\label{eq:1-gamma-theta}
\Gamma^{\theta}_{r\theta}=&\frac{1}{r}\\
\Gamma^{\theta}_{\theta r}=&\frac{1}{r}\\
\Gamma^{\theta}_{\phi\phi}=&-\sin \theta \cos \theta\\
\Gamma^{\theta}_{ij}=&0 \text{ otherwise.}
\end{empheq}
and
\begin{align*}
0=&\,\frac{d}{d \tau}\Big(\frac{\partial L}{\partial \dot{\phi}}\Big)-\frac{\partial L}{\partial \phi}\\
\iff&\\
0=&\frac{d}{d \tau}\Big( 2 r^2 sin^2\theta \dot{\phi}\Big) - 0\\
=&4 r \sin^2 \theta \dot {r} \dot{\phi}+ 4 r^2 sin\theta cos\theta \dot{\theta} \dot{\phi}+2 r^2 sin^2\theta \ddot{\phi}\\
=& 2 r^2 sin^2\theta \Big(\ddot{\phi} + \frac{2}{r}\dot{r}\dot{\phi} + 2 \cot\theta \, \dot{\theta}\dot{\phi}\Big)\text{, whence, from Lemma \ref{lemma:geodesic}}
\end{align*}

\begin{empheq}[left=\empheqlbrace]{align*}\numberthis\label{eq:1-gamma-phi}
\Gamma^{\phi}_{r\phi}=&\frac{1}{r}\\
\Gamma^{\phi}_{\phi r}=&\frac{1}{r}\\
\Gamma^{\phi}_{\theta\phi}=&\cot\theta\\
\Gamma^{\phi}_{\phi\theta}=&\cot\theta\\
\Gamma^{\phi}_{ij}=&0 \text{, otherwise.}
\end{empheq}
Now the Ricci Tensor is given in \cite[II, (58) and (56)]{Akhmedov2017} to be:
\begin{align*}
R_{jl}\equiv&\, R^i_{\,jil}\text{, where}\\
R^i_{\,jkl}\equiv&\,\partial_k\Gamma^i_{jl}-\partial_l\Gamma^i_{jk}+\Gamma^i_{mk} \Gamma^m_{jl} - \Gamma^i_{ml} \Gamma^m_{jk}
\end{align*}

\subsection{Ricci Tensor}
To avoid confusion with indices, we suspend the summation convention until the end of this section.
\begin{align*}
R_{rr}=&\underbrace{R^r_{\,rrr}}_\text{$=0$ from \eqref{eq:as1}}+R^{\theta}_{\,r\theta r}+R^{\,\phi}_{r\phi r}\\
R^{\theta}_{\,r\theta r}=&\underbrace{\partial_{\theta}\Gamma^{\theta}_{rr}}_\text{$=0$ from \eqref{eq:1-gamma-theta}}-\overbrace{\partial_r\Gamma^{\theta}_{r\theta}}^\text{$\frac{-1}{r^2}$, from \eqref{eq:1-gamma-theta}}+\underbrace{\sum_m\Gamma^{\theta}_{m\theta}\Gamma^{m}_{rr}}_\text{$-2 \alpha (1+\alpha r^2)^{-1}$, from \eqref{eq:1-gamma-r} and \eqref{eq:1-gamma-theta}}-\overbrace{\sum_m\Gamma^{\theta}_{mr}\Gamma^m_{r\theta}}^\text{$\frac{1}{r^2}$}\\
=&-2 \alpha (1+\alpha r^2)^{-1}\\
R^{\,\phi}_{r\phi r}=&\underbrace{\partial_{\phi}\Gamma^{\phi}_{rr}}_\text{$=0$, from \eqref{eq:1-gamma-phi}}-\overbrace{\partial_r\Gamma^{\phi}_{r\phi}}^\text{$\frac{-1}{r^2}$, from \eqref{eq:1-gamma-phi}}+\underbrace{\sum_m\Gamma^{\phi}_{m\phi}\Gamma^{m}_{rr}}_\text{$-2 \alpha (1+\alpha r^2)^{-1}$, from \eqref{eq:1-gamma-r} and \eqref{eq:1-gamma-phi}}-\overbrace{\sum_m\Gamma^{\phi}_{mr}\Gamma^m_{r\phi}}^\text{$\frac{1}{r^2}$}\text{, whence:}\\
R_{rr}=&-4 \alpha (1+\alpha r^2)^{-1}\numberthis\label{eq:R-r-r}
\end{align*}

\begin{align*}
R_{\theta\theta}=&R\indices{^r_{\theta r\theta}}+\underbrace{R\indices{^{\theta}_{\theta\theta\theta}}}_{=0}+R\indices{^{\phi}_{\theta\phi\theta}}\\
R\indices{^r_{\theta r\theta}}=&\underbrace{\partial_r\Gamma^r_{\theta\theta}}_{=-(1+\alpha r^2)-2ar^2}-\overbrace{\partial_\theta\Gamma^r_{\theta r}}^{=0}+\underbrace{\Gamma^r_{mr}\Gamma^m_{\theta\theta}}_{=\frac{-2\alpha r}{1+2\alpha r^2}(-r)(1+ar^2)}-\overbrace{\Gamma^r_{m\theta}\Gamma^m_{\theta r}}^{=-r(1+\alpha r^2)\frac{1}{r}}\\
=&-(1+\alpha r^2)-2\alpha r^2 + 2\alpha r^2-(1+\alpha r^2)\\
=&0\\
R\indices{^{\phi}_{\theta\phi\theta}}=&\underbrace{\partial_{\phi}\Gamma^{\phi}_{\theta\theta}}_{=0}-\overbrace{\partial_{\theta}\Gamma^{\phi}_{\theta\phi}}_{\partial_{\theta}\cot\theta}+\underbrace{\sum_m\Gamma^{\phi}_{m\phi}\Gamma^m_{\theta\theta}}_{-\frac{1}{r}r(1+\alpha r^2)}-\overbrace{\sum_m\Gamma^{\phi}_{m\theta}\Gamma^{m}_{\theta\phi}}^{=\cot^2\theta}\\
=&1+\cot^2\theta-(1+\alpha r^2)-\cot^2\theta\\
=&-ar^2\\
R_{\theta\theta} = -\alpha r&^2\numberthis\label{eq:R-theta-theta}
\end{align*}

\begin{align*}
R_{\phi\phi}=& R^r_{\phi r \phi} + R^{\theta}_{\phi\theta\phi}+\underbrace{R^{\phi}_{\phi\phi\phi}}_{=0}\\
R^r_{\phi r \phi}=&\partial_r \Gamma^r_{\phi\phi}-\partial_{\phi}\Gamma^{\phi}_{\phi r} +\sum_m\Gamma^r_{mr}\Gamma^m_{\phi\phi}-\sum_m\Gamma^r_{m\phi}\Gamma^m_{\phi r}\\
=&-(1+\alpha r^2)sin^2 \theta-2\alpha r^2 \sin^2 \theta-0+2\alpha r \cancel{(1+\alpha r^2)}^{-1}r\cancel{(1+\alpha r^2)} sin^2\theta + \cancel{r}(1+\alpha r^2) sin^2\theta\frac{1}{\cancel{r}}\\
=&0\\
R^{\theta}_{\phi\theta\phi}=&\underbrace{\partial_\theta\Gamma^{\theta}_{\phi\phi}}_{=0}-\underbrace{\partial_{\phi}\Gamma^{\theta}_{\phi\theta}}_{=0} + \underbrace{\sum_m\Gamma^{\theta}_{m\theta}\Gamma^m_{\phi\phi}}_{m=r}-\underbrace{\sum_m\Gamma^{\theta}_{m\phi}\Gamma^m_{\phi\theta}}_{m=\phi}\\
=&-\frac{1}{\cancel{r}}\cancel{r}(1+\alpha r^2)\sin^2\theta-0\\
R_{\phi\phi}=& -(1+\alpha r^2)\sin^2\theta \numberthis\label{eq:R-phi-phi}
\end{align*}

\begin{align*}
R_{\theta\phi}=&R^r_{\,\theta r \phi}+\underbrace{R^{\theta}_{\,\theta\theta\phi}}_{=\gamma^{\theta\theta}R_{\theta\theta\theta\phi}=0}+\underbrace{R^{\phi}_{\,\theta\phi\phi}}_{=0}\\
=&\partial_r\Gamma^r_{\theta\phi}-\partial_{\phi}\Gamma^r_{\theta r}+\sum_m\Gamma^r_{mr}\Gamma^m_{\theta\phi}-\sum_m\Gamma^r_{m\phi}\Gamma^m_{\theta r}\\
=&0\\
R_{\phi r}=&\underbrace{R^r_{\,\phi rr}}_{=0} + R^{\theta}_{\,\phi\theta r} + \underbrace{R^{\phi}_{\,\phi\phi r}}_{=0}\\
=&\partial_r \Gamma^r_{\theta\phi} - \partial_{\phi} \Gamma^r_{\theta r} + \sum_m\Gamma^r_{m\theta} \Gamma^m_{\phi r} - \sum_m\Gamma^r_{mr} \Gamma^m_{\phi \theta} \\
=& 0\\
R_{r\theta}=&\underbrace{R^r_{\,rr \theta}}_{=0} + \underbrace{R^{\theta}_{\,r\theta\theta}}_{=0} + R^{\phi}_{\,r\phi\theta}\\
=&\underbrace{\partial_{\phi} \Gamma^{\phi}_{r\theta}}_{=0} - \underbrace{\partial_{\theta} \Gamma^{\phi}_{r\phi}}_{=0} + \underbrace{\sum_m\Gamma^{\phi}_{m\phi} \Gamma^m_{r\theta}}_{m=r,\theta} - \underbrace{\sum_m\Gamma^{\phi}_{m\theta} \Gamma^m_{r\phi}}_{m=\phi} \\
=&\frac{1}{r}0 + \cot\theta\frac{1}{r}-\cot\theta\frac{1}{r}\\
=&0
\end{align*}

So, setting $\alpha=-1$
\begin{empheq}[left=\empheqlbrace]{align*}
R_{rr}=&-4  (1-r^2)^{-1}\numberthis\label{eq:ricci-1}\\
R_{\theta\theta} =& r^2\\
R_{\phi\phi} =& -(1- r^2)\sin^2\theta\\
R_{\theta\phi}=&R_{\phi\theta}=0\\
R_{\phi r}=&R_{r\phi}=0\\
R_{r\theta}=&R_{\theta r}=0
\end{empheq}

\subsection{Einstein Equation}
\begin{empheq}[left=\empheqlbrace]{align*}
\gamma_{11}=&(1-r^2)^{-1}\\
\gamma_{22}=&r^2\\
\gamma_{33}=&r^2\sin^2\theta\\
\gamma_{ij}=&0	
\end{empheq}

\section{Metric: $dl^2\equiv d\chi^2+\sin^2 \chi\,d\Omega^2$} \label{section:metric2}
We will use the well known formula $\sinh(\chi)=-i \sin(i\chi)$, \cite{wiki:sinh}, to reduce the metrics of Sections \ref{section:metric2} and \ref{section:metric4} to a common form.
\subsection{Christoffel Symbols}

\begin{align*}
L=&\dot{\chi}^2 + \alpha \sin^2(\beta \chi)(\dot\theta^2 + \sin^2 \theta \dot \phi^2)\text{, where $\alpha=1,\beta=1$ or $\alpha=-1,\beta=i$} \numberthis \label{eq:lagrangian2}\\
0=&\frac{d}{d \tau}\Big(\frac{\partial L}{\partial \dot{\chi}}\Big)-\frac{\partial L}{\partial \chi}\\
=&\ddot{\chi}-2\alpha\beta \cos{(\beta\chi)}\sin{(\beta\chi)}(\dot\theta^2 + \sin^2 \theta \dot \phi^2)\text{, whence, from Lemma \ref{lemma:geodesic}:}\\
\Gamma^{\chi}_{\theta\theta}=&-2\alpha\beta \cos{(\beta\chi)}\sin{(\beta\chi)}\\
\Gamma^{\chi}_{\phi\phi}=&-2\alpha\beta \cos{(\beta\chi)}\sin{(\beta\chi)}\sin^2 \theta\\
\Gamma^{\chi}_{jk}=&0
\end{align*}

\begin{align*}
0=&\frac{d}{d \tau}\Big(\frac{\partial L}{\partial \dot{\theta}}\Big)-\frac{\partial L}{\partial \theta}\\
=&\frac{d}{d \tau}\big(2\alpha\sin^2{(\beta\chi)}\dot{\theta}\big)-2\alpha\sin^2{(\beta\chi)}\sin{(\theta)}\cos{(\theta)}\dot{\phi}^2\\
=&2\alpha\big[\sin^2{(\beta\chi)}\ddot{\theta}+2\beta\sin{(\beta\chi)}\cos{(\beta\chi)}\dot{\chi}\dot{\theta}-\sin^2{(\beta\chi)}\sin{(\theta)}\cos{(\theta)}\dot{\phi}^2\big]\\
=&2\alpha\sin^2{(\beta\chi)}\big[\ddot{\theta}+2\cot{(\beta\chi)}\dot{\chi}\dot{\theta}-\sin{(\theta)}\cos{(\theta)}\dot{\phi}^2\big]\text{, whence from Lemma \ref{lemma:geodesic}:}\\
\Gamma^{\theta}_{\chi\theta}=&cot{(\beta\chi)}\\
\Gamma^{\theta}_{\theta\chi}=&cot{(\beta\chi)}\\
\Gamma^{\theta}_{\phi\phi}=&-\sin{(\theta)}\cos{(\theta)}\\
\Gamma^{\theta}_{ij}=&0\text{ otherwise}
\end{align*}

\begin{align*}
0=&\frac{d}{d \tau}\Big(\frac{\partial L}{\partial \dot{\phi}}\Big)-\frac{\partial L}{\partial \phi}\\
=&2\frac{d}{d \tau}\big(\alpha \sin^2(\beta \chi) \sin^2 \theta\, \dot \phi\big)\\
=&2\big[\big(\alpha \sin^2(\beta \chi) \sin^2 \theta\, \ddot \phi\big) + \big(2 \alpha \beta \sin(\beta \chi) \cos(\beta \chi) \sin^2 \theta \dot{\chi}\dot \phi\big) + \big(2\alpha \sin^2(\beta \chi) \sin \theta \cos\theta \dot\theta \dot \phi\big)\big]\\
=&2\alpha \sin^2(\beta \chi) \sin^2 \theta\big[\ddot{\phi}+2 \beta \cot{(\beta\chi)} \dot{\chi}\dot{\phi}+2\cot{\theta\dot\theta \dot \phi}\big]\text{, whence from Lemma \ref{lemma:geodesic}:}\\
\Gamma^{\phi}_{\theta\chi}=&\beta\cot{(\beta\chi)}\\
\Gamma^{\phi}_{\chi\theta}=&\beta\cot{(\beta\chi)}\\
\Gamma^{\phi}_{\theta\phi}=&cot{(\theta)}\\
\Gamma^{\phi}_{\theta\phi}=&cot{(\theta)}\\
\Gamma^{\phi}_{ij}=&0\text{, otherwise}
\end{align*}
From \eqref{eq:lagrangian2}

\begin{align*}
|g|=\alpha^2sin^4{(\beta\chi)}sin^2{(\theta)}
\end{align*}

So setting $\alpha=1,\beta=1$ 


\section{Metric: $dl^2\equiv\frac{dr^2}{1+r^2}+r^2\,d\Omega^2$}  \label{section:metric3}
So, setting $\alpha=+1$ in \eqref{eq:R-r-r}, \eqref{eq:R-theta-theta}
\begin{empheq}[left=\empheqlbrace]{align*}
R_{rr}=&4  (1+r^2)^{-1}\\
R_{\theta\theta} =& - r^2
\end{empheq}

\section{Metric: $dl^2\equiv d\chi^2+\sinh^2 \chi\,d\Omega^2$} \label{section:metric4}

So, setting $\alpha=-1,\beta=i$
\begin{thebibliography}{9}\label{section:biblio}
	\raggedright
	\bibitem{Akhmedov2017}
	Emil T. Akhmedov,
	\emph{Lectures on General Theory of Relativity},
	arXiv:1601.04996,
	\url{https://arxiv.org/abs/1601.04996}
	\bibitem{abs1965}
	Ronald Adler, Maurice Bazin, \& Menahem Schiffer,
	\emph{Introduction to General Relativity},
	McGraw-Hill Book Company, New York,
	1965.
	\bibitem{wiki:sinh}
	Wikipedia contributors
	\emph{Hyperbolic function. Wikipedia, The Free Encyclopedia. September 25, 2017, 01:29 UTC.}
	 Available at: 
	 \url{https://en.wikipedia.org/w/index.php?title=Hyperbolic_function&oldid=802265878}. Accessed October 21, 2017
\end{thebibliography}

\end{document}
