\documentclass[]{article}
\usepackage{mathtools,amssymb,amsthm,url,cancel}
\newtheorem{theorem}{Theorem}
\newcommand\numberthis{\addtocounter{equation}{1}\tag{\theequation}}

% Title Page
\title{Introduction into General Relativity\\Assignment 9.2\\Gravitational Waves}
\author{Simon Crase}


\begin{document}
\maketitle


\begin{theorem}
Consider the metric tensor of the form: $g_{\mu\nu} \approx \eta_{\mu\nu} + h_{\mu\nu}$,where |$h_{\mu\nu}| \ll 1.$ Here $\eta_{\mu\nu}$ is the background Minkowskian metric tensor, while $h_{\mu\nu}$ is a small perturbation on top of it.

Then that the Riemann tensor has the following form at the linear order in $h_{\mu\nu}$:

\begin{align*}
R^{(1)}_{\mu\nu\alpha\beta} =& \eta_{\mu\gamma}\big[\partial_{\alpha}\Gamma^{\gamma}_{\nu\beta} - \partial_{\beta}\Gamma^{\gamma}_{\nu\alpha}\big]\\
\approx& \frac{1}{2}\big[\partial_{\nu}\partial_{\alpha}h_{\mu\beta} + \partial_{\mu}\partial_{\beta}h_{\nu\alpha} - \partial_{\mu}\partial_{\alpha}h_{\nu\beta} - \partial_{\nu}\partial_{\beta}h_{\mu\alpha}\big]
\end{align*}

Furthermore, in the gauge 
\begin{align*}
\partial_{\mu}(h^{\mu}_{\nu}-\frac{1}{2}\delta^{\mu}_{\nu}h)=&0\text{, the Ricci tensor is:}\\
R^{(1)}_{\mu\nu} =& - \frac{1}{2} \Box h_{\mu_{\nu}} \text{. where:}\\
\Box \triangleq & \eta^{\alpha\beta}\partial_{\alpha}\partial_{\beta}
\end{align*}
 

\end{theorem}


\begin{thebibliography}{9}
	
	\bibitem{akhmedev2016}
	Emil T. Akhmedev,
	\emph{Lectures on General Theory of Relativity},
	2016,
	\url{https://arxiv.org/pdf/1601.04996v6.pdf}.
	
	
\end{thebibliography}

\end{document}          
