\documentclass[]{article}
\usepackage{mathtools,amssymb,amsthm,url,cancel}

%opening
\title{Introduction into General Relativity\\Assignment 8\\The d'Alembert operator}
\author{Simon Crase}

\newtheorem{theorem}{Theorem}
\newtheorem{lemma}[theorem]{Lemma}
\newtheorem{remark}[theorem]{Remark}
\newcommand\numberthis{\addtocounter{equation}{1}\tag{\theequation}}

\begin{document}

\maketitle


\begin{theorem}
\begin{align*}
	\hbar^2 \Box e^{-i\frac{\Psi}{\hbar}}=&0 \numberthis \label{eq:hypothesis}\\ \text{ where } \Box \triangleq& \frac{1}{\sqrt(|g|)} \partial_{\mu} g^{\mu\nu}\sqrt(|g|)\partial_{\nu}\numberthis\label{eq:dlambert}\\ \Rightarrow&\\ g^{\mu\nu}\partial_{\mu}\Psi \partial_{\nu}\Psi=&0 \text{ as }\hbar \rightarrow 0.
\end{align*}
\end{theorem}

\begin{proof}
\begin{align*}
	\text{(\ref{eq:hypothesis}) \& (\ref{eq:dlambert})} \Rightarrow&\\
	\hbar^2& \frac{1}{\sqrt(|g|)} \partial_{\mu}\big[ g^{\mu\nu}\sqrt(|g|)\partial_{\nu}\big(e^{-i\frac{\Psi}{\hbar}}\big)\big]=0\\
	\Rightarrow&\\
	\hbar^2& \frac{1}{\sqrt(|g|)} \partial_{\mu}\big[ g^{\mu\nu}\sqrt(|g|)e^{-i\frac{\Psi}{\hbar}}\big(-\frac{i}{\hbar}\partial_{\nu}\Psi\big)\big]=0\\
	\Rightarrow&\\
	-i \hbar& \frac{1}{\sqrt(|g|)} \partial_{\mu}\big[ g^{\mu\nu}\sqrt(|g|)e^{-i\frac{\Psi}{\hbar}}\big(\partial_{\nu}\Psi\big)\big]=0\\
	\Rightarrow&\\
	-i \hbar& \frac{1}{\sqrt(|g|)} \partial_{\mu}\bigg[\big[g^{\mu\nu}\sqrt(|g|)\big(\partial_{\nu}\Psi\big)\big] e^{-i\frac{\Psi}{\hbar}}\bigg]=0 \\
	\Rightarrow& \text{ \{by Leibniz' rule\}}\\
	-i \hbar& \frac{1}{\sqrt(|g|)} \bigg[\partial_{\mu}\big[g^{\mu\nu}\sqrt(|g|)\big(\partial_{\nu}\Psi\big)\big] e^{-i\frac{\Psi}{\hbar}} + \big[g^{\mu\nu}\sqrt(|g|)\big(\partial_{\nu}\Psi\big)\big] \partial_{\mu}e^{-i\frac{\Psi}{\hbar}}\bigg]=0 \\
	\Rightarrow&\\
	-i \hbar& \frac{1}{\sqrt(|g|)} \bigg[\partial_{\mu}\big[g^{\mu\nu}\sqrt(|g|)\big(\partial_{\nu}\Psi\big)\big]  + \big[g^{\mu\nu}\sqrt(|g|)\big(\partial_{\nu}\Psi\big)\big](\frac{-i}{\hbar}) \partial_{\mu}\Psi\bigg]e^{-i\frac{\Psi}{\hbar}}=0
\end{align*}	

Now $e^{-i\frac{\Psi}{\hbar}}\neq0$, hence the foregoing simplifies to:
\begin{align*}
		\text{(\ref{eq:hypothesis}) \& (\ref{eq:dlambert})} \Rightarrow&\\
			-i \hbar& \frac{1}{\sqrt(|g|)} \bigg[\partial_{\mu}\big[g^{\mu\nu}\sqrt(|g|)\big(\partial_{\nu}\Psi\big)\big]  + \big[g^{\mu\nu}\sqrt(|g|)\big(\partial_{\nu}\Psi\big)\big](\frac{-i}{\hbar}) \partial_{\mu}\Psi\bigg]=0\\
			 \Rightarrow& \text{\{since $i\neq0$\}}\\
			\hbar& \frac{1}{\sqrt(|g|)} \bigg[\partial_{\mu}\big[g^{\mu\nu}\sqrt(|g|)\big(\partial_{\nu}\Psi\big)\big]  + \big[g^{\mu\nu}\sqrt(|g|)\big(\partial_{\nu}\Psi\big)\big](\frac{-i}{\hbar}) \partial_{\mu}\Psi\bigg]=0 \\
			\Rightarrow&\\
			 &\hbar \frac{1}{\sqrt(|g|)}\partial_{\mu}\big[g^{\mu\nu}\sqrt(|g|)\big(\partial_{\nu}\Psi\big)\big]  + \cancel{\hbar} \frac{1}{\sqrt(|g|)}\big[g^{\mu\nu}\sqrt(|g|)\big(\partial_{\nu}\Psi\big)\big](\frac{-i}{\cancel{\hbar}}) \partial_{\mu}\Psi=0 \\
			 \Rightarrow&\\
			 &\hbar \frac{1}{\sqrt(|g|)}\partial_{\mu}\big[g^{\mu\nu}\sqrt(|g|)\big(\partial_{\nu}\Psi\big)\big]  -i  \frac{1}{\cancel{\sqrt(|g|)}}\big[g^{\mu\nu}\cancel{\sqrt(|g|)}\big(\partial_{\nu}\Psi\big)\big] \partial_{\mu}\Psi=0\\
			 \Rightarrow&\\
			 &\hbar \frac{1}{\sqrt(|g|)}\partial_{\mu}\big[g^{\mu\nu}\sqrt(|g|)\big(\partial_{\nu}\Psi\big)\big]  -i  \big[g^{\mu\nu}\big(\partial_{\nu}\Psi\big)\big] \partial_{\mu}\Psi=0 \numberthis\label{eq:2nd}
\end{align*}
If we now let $\hbar \rightarrow 0$, (\ref{eq:2nd}), becomes:
\begin{align*}
	\text{(\ref{eq:hypothesis}) \& (\ref{eq:dlambert}) \& } \hbar \rightarrow 0 \Rightarrow&\\
	-i  g^{\mu\nu}\partial_{\nu}\Psi \partial_{\mu}\Psi\rightarrow&0\\
	 \Rightarrow&\\
	g^{\mu\nu}\partial_{\nu}\Psi \partial_{\mu}\Psi\rightarrow&0 
\end{align*}
\end{proof}

\begin{thebibliography}{9}
	
	\bibitem{akhmedev2016}
	Emil T. Akhmedev,
	\emph{Lectures on General Theory of Relativity},
	2016,
	\url{https://arxiv.org/pdf/1601.04996v6.pdf}.
	

\end{thebibliography}

\end{document}
