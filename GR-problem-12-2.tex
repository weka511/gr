\documentclass[]{article}
\usepackage{esvect,amsmath,url,amsthm,cancel}
\newtheorem{theorem}{Theorem}
\newcommand\numberthis{\addtocounter{equation}{1}\tag{\theequation}}
%opening
\title{Introduction into General Relativity\\Assignment 12.2\\Cosmological solutions with $\Lambda\ne 0$}
\author{Simon Crase}

\begin{document}

\maketitle

\begin{abstract}
	We derive the metric for the Poincar\'e Patch, \cite[XII,(332)]{Akhmedov2017}, from the transformation \cite[XII,(333)]{Akhmedov2017}.
\end{abstract}

\begin{theorem}
	\begin{align*}
	HX^0=&\sinh(H\tau_+)+\frac{(H\vec{x}_+)^2}{2}e^{H\tau_+}\text{, \cite[XII,(333)]{Akhmedov2017}}\numberthis\label{eq:x0}\\
	HX^i=&Hx^i_+e^{H\tau_+}, \vec{x}++\equiv x+^i_+, i=1,...,D-1\numberthis\label{eq:xi}\\
	HX^D=&-\cosh(H\tau_+)+\frac{(H\vec{x}_+)^2}{2}e^{H\tau_+}\numberthis\label{eq:xD}\\
	ds^2=&\eta_{AB}dx^Adx^B\numberthis\label{eq:ds}\\
	\implies&\\
	ds^2_+=&d\tau^2_+ - e^{2H\tau_+}d\vec{x}^2_+\text{, \cite[XII,(332)]{Akhmedov2017}}	\numberthis\label{eq:demonstandem}
	\end{align*}
\end{theorem}
\begin{proof}
	Since \eqref{eq:ds} shows that $ds^2$ depends on the $\{dX^{\mu}\}$, we differentiate the other three equations.
	\begin{align*}
	\text{\eqref{eq:x0}}\implies dX^0=&\cosh(H\tau_+)d\tau_+ +H\frac{d(\vec{x}_+)^2}{2}e^{H\tau_+}+H^2\frac{(\vec{x}_+)^2}{2}e^{H\tau_+}d\tau_+\\
	=&\cosh(H\tau_+)d\tau_+ +H\vec{x}_+\bullet d(\vec{x}_+)e^{H\tau_+}+H^2\frac{(\vec{x}_+)^2}{2}e^{H\tau_+}d\tau_+\numberthis\label{eq:dx0}\\
	\eqref{eq:xi}\implies	dX^i=&dx^i_+e^{H\tau_+} + H x^i_+e^{H\tau_+}d\tau_+\numberthis\label{eq:dxi}\\
	\eqref{eq:xD}\implies dX^D=&-\sinh(H\tau_+)d\tau_+ +H\vec{x}_+\bullet d(\vec{x}_+)e^{H\tau_+} +H^2 \frac{(\vec{x}_+)^2}{2}e^{H\tau_+}d\tau_+ \numberthis\label{eq:dxD}
	\end{align*}
	From \eqref{eq:ds}
	\begin{align*}
	ds^2=&\eta_{AB}dx^Adx^B\\
	=&(dX^0)^2- \sum_{i=1}^{D-1}(dX^i)^2-(dX^D)^2\\
	=&\bigg[\cosh(H\tau_+)d\tau_+ +\big(H\vec{x}_+\bullet d(\vec{x}_+)e^{H\tau_+}+H^2\frac{(\vec{x}_+)^2}{2}e^{H\tau_+}d\tau_+\big)\bigg]^2\\
	&- \sum_{i=1}^{D-1}\bigg[dx^i_+e^{H\tau_+} + H x^i_+e^{H\tau_+}d\tau_+\bigg]^2\\
	&-\bigg[-\sinh(H\tau_+)d\tau_+ +\big(H\vec{x}_+\bullet d(\vec{x}_+)e^{H\tau_+} +H^2\frac{(H\vec{x}_+)^2}{2}e^{H\tau_+}d\tau_+\big)\bigg]^2\\
	=&\cosh^2(H\tau_+)d\tau_+^2+\cancel{\big(H\vec{x}_+\bullet d(\vec{x}_+)^2e^{2H\tau_+}}+\bcancel{H^4\frac{(\vec{x}_+)^4}{4}e^{2H\tau_+}d\tau_+^2}\\
	&+2\big(H\vec{x}_+\bullet d(\vec{x}_+)e^{H\tau_+} H^2\frac{(H\vec{x}_+)^2}{2}e^{H\tau_+}d\tau_+\\
	&+2\cosh(H\tau_+)d\tau_+ H^2\frac{(\vec{x}_+)^2}{2}e^{H\tau_+}d\tau_+\big)\\
	&+2\cosh(H\tau_+)d\tau_+ \big(H\vec{x}_+\bullet d(\vec{x}_+)e^{H\tau_+}\big)\\
	&- \sum_{i=1}^{D-1}(dx^i_+)^2e^{2H\tau_+} \\
	&- 2\sum_{i=1}^{D-1}dx^i_+e^{H\tau_+}  H x^i_+e^{H\tau_+}d\tau_+\\
	&- \sum_{i=1}^{D-1}\bigg[ H x^i_+e^{H\tau_+}d\tau_+\bigg]^2\\
	&-\sinh^2(H\tau_+)d\tau_+^2-\cancel{\big(H\vec{x}_+\bullet d(\vec{x}_+)^2e^{2H\tau_+}}-\bcancel{H^4\frac{(\vec{x}_+)^4}{4}e^{2H\tau_+}d\tau_+^2}\\
	&-2\big(H\vec{x}_+\bullet d(\vec{x}_+)e^{H\tau_+} H^2\frac{(H\vec{x}_+)^2}{2}e^{H\tau_+}d\tau_+\big)\\
	&+2\sinh(H\tau_+)d\tau_+ H^2\frac{(H\vec{x}_+)^2}{2}e^{H\tau_+}d\tau_+\\
	&+2\sinh(H\tau_+)d\tau_+ H\vec{x}_+\bullet d(\vec{x}_+)e^{H\tau_+} 	
	\end{align*}
	We group terms as shown.
	\begin{align*}
	ds^2=\big[&\cosh^2(H\tau_+)\\
	&+ 2\cosh(H\tau_+) H^2\frac{(\vec{x}_+)^2}{2}e^{H\tau_+}\\
	&-\sinh^2(H\tau_+)\\
	&- \sum_{i=1}^{D-1}\bigg[ H x^i_+e^{H\tau_+}\bigg]^2\\
	&+2\sinh(H\tau_+) H^2\frac{(H\vec{x}_+)^2}{2}e^{H\tau_+}\\	
	&\big] d\tau_+^2\\
	+\big[&2\big(e^{H\tau_+} H^2\frac{(H\vec{x}_+)^2}{2}e^{H\tau_+}\\
	&+2\cosh(H\tau_+) e^{H\tau_+}\\
	&- 2e^{H\tau_+}  e^{H\tau_+}\text{, since $\sum_{i=1}^{D-1}dx^i_+ H x^i_+=H\vec{x}_+\bullet d(\vec{x}_+)$}\\
	&-2\big(e^{H\tau_+} H^2\frac{(H\vec{x}_+)^2}{2}e^{H\tau_+}\big)\\
	&+2\sinh(H\tau_+)e^{H\tau_+}\\
	\big]&H\vec{x}_+\bullet d(\vec{x}_+)d\tau_+\\
	-& e^{2H\tau_+}\underbrace{\sum_{i=1}^{D-1}(dx^i_+)^2}_\text{$=d\vec{x}^2_+$}\numberthis\label{eq:expanded}
	\end{align*}
	Rearranging the $d\tau_+^2$ term in \eqref{eq:expanded} gives:
	\begin{align*}
	&\cosh^2(H\tau_+)-\sinh^2(H\tau_+)\\
	&+  H^2\, \cancel{2}\cosh(H\tau_+)\frac{(\vec{x}_+)^2}{\cancel{2}}e^{H\tau_+}+H^2\, \cancel{2}\sinh(H\tau_+) \frac{(\vec{x}_+)^2}{\cancel{2}}e^{H\tau_+}\\
	&- \sum_{i=1}^{D-1}\bigg[ H x^i_+e^{H\tau_+}\bigg]^2\\
	=& 1 +H^2\frac{e^{H\tau_+}+\cancel{e^{-H\tau_+}}+e^{H\tau_+}-\cancel{e^{-H\tau_+}}}{2}(\vec{x}_+)^2e^{H\tau_+} - H^2e^{2H\tau_+}\sum_{i=1}^{D-1}\bigg[  x^i_+ \bigg]^2\\
	=& 1 + \cancel{H^2 (\vec{x}_+)^2 2e^{H\tau_+}} - \cancel{H^2 (\vec{x}_+)^2 2e^{H\tau_+}}\\
	=&1 \numberthis\label{eq:d_tau_2}
	\end{align*}
	
	Rearranging the $H\vec{x}_+\bullet d(\vec{x}_+)d\tau_+$ term in \eqref{eq:expanded} gives:
	\begin{align*}
&\cancel{2e^{H\tau_+} H^2\frac{(H\vec{x}_+)^2}{2}e^{H\tau_+}}+2\cosh(H\tau_+) - 2e^{H\tau_+}  e^{H\tau_+}\\
&-\cancel{2e^{H\tau_+} H^2\frac{(H\vec{x}_+)^2}{2}e^{H\tau_+}}+2\sinh(H\tau_+)e^{H\tau_+}\\
=&2\big[\underbrace{\cosh(H\tau_+)+\sinh(H\tau_+)}_\text{$=\frac{e^{H\tau_+}+\cancel{e^{-H\tau_+}}+e^{H\tau_+}-\cancel{e^{-H\tau_+}}}{2}$}-e^{H\tau_+}\big]e^{H\tau_+}\\
=&2\big[e^{H\tau_+}-e^{H\tau_+}\big]e^{H\tau_+}\\
=&0\numberthis\label{eq:d_mixed}
	\end{align*}
Substituting \eqref{eq:d_tau_2} and \eqref{eq:d_mixed} in \eqref{eq:expanded} yields $ds^2_+=d\tau^2_+ - e^{2H\tau_+}d\vec{x}^2_+$, which matches \eqref{eq:demonstandem}.
\end{proof}

\begin{thebibliography}{9}\label{section:biblio}
	\raggedright
	\bibitem{Akhmedov2017}
	Emil T. Akhmedov,
	\emph{Lectures on General Theory of Relativity},
	arXiv:1601.04996,
	\url{https://arxiv.org/abs/1601.04996}
\end{thebibliography}

\end{document}
