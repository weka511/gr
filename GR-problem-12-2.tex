\documentclass[]{article}
\usepackage{esvect,amsmath,url,amsthm}
\newtheorem{theorem}{Theorem}
\newcommand\numberthis{\addtocounter{equation}{1}\tag{\theequation}}
%opening
\title{Introduction into General Relativity\\Assignment 12.2\\Cosmological solutions with non-zero cosmological constant}
\author{Simon Crase}

\begin{document}

\maketitle

\begin{abstract}
	Derive \cite[XII,(332)]{Akhmedov2017} from \cite[XII,(333)]{Akhmedov2017}
	\begin{itemize}
		\item     Correct calculation of the differentials of the coordinates are given
		\item Correct expression for the induced metric is given
	\end{itemize}
\end{abstract}

\begin{theorem}
	\begin{align*}
	HX^0=&\sinh(H\tau_+)+\frac{(H\vec{x}_+)^2}{2}e^{H\tau_+}\numberthis\label{eq:x0}\\
	HX^i=&Hx^i_+e^{H\tau_+}, \vec{x}++\equiv x+^i_+, i=1,...,D-1\numberthis\label{eq:xi}\\
	HX^D=&-\cosh(H\tau_+)+\frac{(H\vec{x}_+)^2}{2}e^{H\tau_+}\numberthis\label{eq:xD}\\
	ds^2=&\eta_{AB}dx^Adx^B\numberthis\label{eq:ds}\\
	\implies&\\
	ds^2_+=&d\tau^2_+ - e^{2H\tau_+}\vec{x}^2_+	
	\end{align*}
\end{theorem}
\begin{proof}
	Since \eqref{eq:ds} shows that $ds^2$ depends on the $\{dX^{\mu}\}$, we differentiate the other three equations.
	\begin{align*}
	\text{\eqref{eq:x0}}\implies dX^0=&\cosh(H\tau_+)d\tau_+ +H\frac{d(\vec{x}_+)^2}{2}e^{H\tau_+}+H^2\frac{(\vec{x}_+)^2}{2}e^{H\tau_+}d\tau_+\\
	=&\cosh(H\tau_+)d\tau_+ +H\vec{x}_+\bullet d(\vec{x}_+)e^{H\tau_+}+H^2\frac{(\vec{x}_+)^2}{2}e^{H\tau_+}d\tau_+\numberthis\label{eq:dx0}\\
	\eqref{eq:xi}\implies	dX^i=&dx^i_+e^{H\tau_+} + H x^i_+e^{H\tau_+}d\tau_+\numberthis\label{eq:dxi}\\
	\eqref{eq:xD}\implies dX^D=&-\sinh(H\tau_+)d\tau_+ +H\vec{x}_+\bullet d(\vec{x}_+)e^{H\tau_+} +\frac{(H\vec{x}_+)^2}{2}e^{H\tau_+}d\tau_+\ \numberthis\label{eq:dxD}
	\end{align*}
\end{proof}
\begin{thebibliography}{9}\label{section:biblio}
	\raggedright
	\bibitem{Akhmedov2017}
	Emil T. Akhmedov,
	\emph{Lectures on General Theory of Relativity},
	arXiv:1601.04996,
	\url{https://arxiv.org/abs/1601.04996}
\end{thebibliography}

\end{document}
