\documentclass[]{article}
\usepackage{tikz,url,float}
\usetikzlibrary{decorations.pathmorphing}
\usetikzlibrary{decorations.pathmorphing}
\usetikzlibrary{arrows.meta}
\tikzset{>={Latex[width=2mm,length=2mm]}}

%opening
\title{Introduction into General Relativity\\Assignment 5.2\\The black hole Penrose-–Carter diagram\\and Eddington–-Finkelstein coordinates}
\author{Simon Crase}

\begin{document}

\maketitle


\section{Penrose-Carter diagram for Black Hole}
Following the method sketched in \cite{diaz2013}, and using the notation of \cite{frolov2011}, we obtain the Penrose-Carter diagram Figure \ref{fig:pc}.
\begin{figure}[H]
	\begin{tikzpicture}
	\node (I)    at ( 4,0)   {$R_+$};
	\node (II)   at (-4,0)   {$R_-$};
	\node (III)  at (0, 2.5) {$T_-$};
	\node (IV)   at (0,-2.5) {$T_+$};
	
	\path  % Four corners of left diamond
	(II) +(90:4)  coordinate[label=90:$I^+$]  (IItop)
	+(-90:4) coordinate[label=-90:$I^-$] (IIbot)
	+(0:4)   coordinate                  (IIright)
	+(180:4) coordinate[label=180:$I^0$] (IIleft)
	;
	\path
	(I) +(90:4)  coordinate[label=90:$I^+$]  (Itop)
	+(-90:4) coordinate[label=-90:$I^-$] (Ibot)
	+(0:4)   coordinate [label=180:$I^0$]  (Iright)
	+(180:4) coordinate (Ileft)
	;
	
	\draw[dashed,black, thin,->] (7,-1) -- (2,4);
	\draw[dashed,black, thin,->] (3,-3) -- (7,1);
	\draw (IIleft) -- 
	node[midway, above left]    {$\cal{J}^+$}
	(IItop) --
	node[midway, below, sloped] {$r=r_g$}
	(IIright) -- 
	node[midway, below, sloped] {$r=r_g$}
	(IIbot) --
	node[midway, below left]    {$\cal{J}^-$}    
	(IIleft) -- cycle;
	
	\draw (Ileft) -- 
	node[midway, above left,sloped]    {$r=r_g$}
	(Itop) --
	node[midway, below, ] {$\cal{J}^+$}
	(Iright) -- 
	node[midway, below, ] {$\cal{J}^-$}
	(Ibot) --
	node[midway, below left,sloped]     {$r=r_g$}    
	(Ileft) -- cycle;
	

	% No text this time in the next diagram
	\draw  (Ileft) -- (Itop) -- (Iright) -- (Ibot) -- (Ileft) -- cycle;
	
	% Squiggly lines
	\draw[decorate,decoration=zigzag] (IItop) -- (Itop)
	node[midway, above, inner sep=2mm] {$r=0$};
	
	\draw[decorate,decoration=zigzag] (IIbot) -- (Ibot)
	node[midway, below, inner sep=2mm] {$r=0$};
	
	\end{tikzpicture}
	\caption{Penrose-Carter Diagram for Black Hole}
	\label{fig:pc}
\end{figure}

The four labelled areas in Figure \ref{fig:pc} are as follows:
\begin{itemize}
	\item $T_-$  Black Hole Interior;
	\item $R_+$  Black Hole Exterior;
	\item $T_+$  White Hole Interior;
	\item $R_-$  White Hole Exterior.
\end{itemize}
\section{Ingoing Eddington-–Finkelstein coordinates}
The Ingoing Eddington-–Finkelstein coordinates cover $R_+\cup T_-$ in Figure \ref{fig:pc_in}.
\begin{enumerate}
	\item We saw in the Lecture, \cite[IV,Figure 8]{akhmedev2016} that the Ingoing Eddington–-Finkelstein coordinates cover the space outside $r=r_g$, and the space inside (the black hole), with the exception of the singularity at $r=0$.
	\item We also saw, \cite[V,Figure 11]{akhmedev2016} that the space outside corresponds to Figure \ref{fig:pc}[$R_+$], and the space inside to Figure \ref{fig:pc}[$T_-$].
	\begin{enumerate}
		\item For there are light rays running through $R_+$, some going to infinity, others crossing $r=r_g$ into $T_-$
		\item There are no light rays crossing from $T_-$ to $R_+$. Instead they all cross $r=0$.
		\item In short, $T_-$ and $R_+$ behave exactly like the two regions in \cite[V,Figure 11]{akhmedev2016}
	\end{enumerate}
\begin{figure}[H]
	\begin{tikzpicture}
	\node (I)    at ( 4,0)   {$R_+$};
	\node (II)   at (-4,0)   {};
	\node (III)  at (0, 2.5) {$T_-$};
	\node (IV)   at (0,-2.5) {};
	
	\path  % Four corners of left diamond
	(II) +(90:4)  coordinate[label=90:$I^+$]  (IItop)
	+(-90:4) coordinate[label=-90:$I^-$] (IIbot)
	+(0:4)   coordinate                  (IIright)
	+(180:4) coordinate[label=180:$$] (IIleft)
	;
	\path
	(I) +(90:4)  coordinate[label=90:$I^+$]  (Itop)
	+(-90:4) coordinate[label=-90:$I^-$] (Ibot)
	+(0:4)   coordinate [label=180:$I^0$]  (Iright)
	+(180:4) coordinate (Ileft)
	;
	
	\draw (IIleft) -- 
	node[midway, above left]    {}
	(IItop) --
	node[midway, below, sloped] {$r=r_g$}
	(IIright) -- 
	node[midway, below, sloped] {$r=r_g$}
	(IIbot) --
	node[midway, below left]    {}    
	(IIleft) -- cycle;
	
	\draw (Ileft) -- 
	node[midway, above left,sloped]    {$r=r_g$}
	(Itop) --
	node[midway, below, ] {$\cal{J}^+$}
	(Iright) -- 
	node[midway, below, ] {$\cal{J}^-$}
	(Ibot) --
	node[midway, below left,sloped]     {$r=r_g$}    
	(Ileft) -- cycle;
	
	
	% No text this time in the next diagram
	\draw  (Ileft) -- (Itop) -- (Iright) -- (Ibot) -- (Ileft) -- cycle;
	
	% Squiggly lines
	\draw[decorate,decoration=zigzag] (IItop) -- (Itop)
	node[midway, above, inner sep=2mm] {$r=0$};
	
	\draw[decorate,decoration=zigzag] (IIbot) -- (Ibot)
	node[midway, below, inner sep=2mm] {$r=0$};
	
	\end{tikzpicture}
	\caption{Ingoing Eddington–-Finkelstein coordinates}
	\label{fig:pc_in}
	\end{figure}
\end{enumerate}

\section{Outgoing Eddington–-Finkelstein coordinates}
The Outgoing Eddington-–Finkelstein coordinates cover $R_- \cup T_+$ in Figure \ref{fig:pc_out}.
\begin{enumerate}
	\item For the transformation $t\rightarrow -t$ transforms from Ingoing Eddington–-Finkelstein coordinates to Outgoing.
	\item It also reflects \ref{fig:pc} about its horizontal axis, interchanging $T_-$ and $T_+$.
\end{enumerate}
\begin{figure}[H]
\begin{tikzpicture}
\node (I)    at ( 4,0)   {};
\node (II)   at (-4,0)   {$R_-$};
\node (III)  at (0, 2.5) {$T_-$};
\node (IV)   at (0,-2.5) {};

\path  % Four corners of left diamond
(II) +(90:4)  coordinate[label=90:$I^+$]  (IItop)
+(-90:4) coordinate[label=-90:$I^-$] (IIbot)
+(0:4)   coordinate                  (IIright)
+(180:4) coordinate[label=180:$I^0$] (IIleft)
;
\path
(I) +(90:4)  coordinate[label=90:$I^+$]  (Itop)
+(-90:4) coordinate[label=-90:$I^-$] (Ibot)
+(0:4)   coordinate [label=180:$$]  (Iright)
+(180:4) coordinate (Ileft)
;

\draw (IIleft) -- 
node[midway, above left]    {$\cal{J}^+$}
(IItop) --
node[midway, below, sloped] {$r=r_g$}
(IIright) -- 
node[midway, below, sloped] {$r=r_g$}
(IIbot) --
node[midway, below left]    {$\cal{J}^-$}    
(IIleft) -- cycle;

\draw (Ileft) -- 
node[midway, above left,sloped]    {$r=r_g$}
(Itop) --
node[midway, below, ] {}
(Iright) -- 
node[midway, below, ] {}
(Ibot) --
node[midway, below left,sloped]     {$r=r_g$}    
(Ileft) -- cycle;


% No text this time in the next diagram
\draw  (Ileft) -- (Itop) -- (Iright) -- (Ibot) -- (Ileft) -- cycle;

% Squiggly lines
\draw[decorate,decoration=zigzag] (IItop) -- (Itop)
node[midway, above, inner sep=2mm] {$r=0$};

\draw[decorate,decoration=zigzag] (IIbot) -- (Ibot)
node[midway, below, inner sep=2mm] {$r=0$};

\end{tikzpicture}
\caption{Penrose-Carter Diagram Outgoing Eddington–-Finkelstein coordinates}
\label{fig:pc_out}
\end{figure}

\begin{thebibliography}{9}
	\raggedright	
	\bibitem{akhmedev2016}
	Emil T. Akhmedev,
	\emph{Lectures on General Theory of Relativity},
	2016,
	\url{https://arxiv.org/pdf/1601.04996v6.pdf}.
	
	\bibitem{frolov2011}
	Valeri P. Frolov and Andrei Zelnikov
	\emph{Introduction to Black Hole Physics},
	2011,
	\url{https://www.amazon.com/gp/search?index=books&linkCode=qs&keywords=9780191003226}
	
	\bibitem{diaz2013}
	J L Diaz,
	\emph{How to Draw Penrose Diagrams with TikZ},
	2013,
	\url{http://tex.stackexchange.com/questions/99124/how-to-draw-penrose-diagrams-with-tikz}
\end{thebibliography}

\end{document}
