\documentclass[]{article}

\usepackage{mathtools,mathrsfs,url,float,amssymb,amsthm,eufrak}


%opening
\title{Introduction into General Relativity\\Assignment 7\\The normal to the surface $r=r_+$\\is light--like}
\author{Simon Crase}

\newtheorem{theorem}{Theorem}
\newtheorem{lemma}[theorem]{Lemma}
\newtheorem{remark}[theorem]{Remark}
\newcommand\numberthis{\addtocounter{equation}{1}\tag{\theequation}}
\begin{document}

\maketitle

\begin{theorem}
	For the Kerr solution, the normal vector to the surface $r=r_+$ is light--like.
\end{theorem}

\begin{proof}
	From \cite[Lecture VII,5,(174)]{akhmedev2016}, the metric is given by the following equations.
	\begin{align*}
	ds^2 = \frac{\Delta-a^2 (\sin{\theta})^2}{\rho^2}dt^2 +& \frac{4 \kappa M a}{\rho^2}r(\sin{\theta})^2 d\phi dt - \frac{\rho^2}{\Delta} dr^2\\
	-&\rho^2 d\theta^2 - \frac{A (\sin{\theta})^2}{\rho^2} d\phi^2 \numberthis \label{eq:metric}\\ 
	\text{where } \Delta =& r^2 - 2 \kappa M + a^2\\
	\rho^2 =& r^2 + a^2 (\cos{\theta})^2\\
	A =& (r^2+a^2)^2 - a^2 \Delta (\sin{\theta})^2 \\
	\text{Moreover } r_{\pm} =& \kappa M \pm \sqrt{(\kappa M)^2 - a^2} \\
	\implies& \Delta(r_{\pm})= 0\numberthis \label{eq:rpm}
	\end{align*}
	
	Now a unit normal to any surface of constant $r$ is given by the covariant vector $\mathfrak{n}=(0,1,0,0)$, and we know that $\mathfrak{n}$ \emph{is light-like} $\iff n_{\mu}n^{\mu}=0 \iff g^{\mu\nu}n_{\mu}n_{\nu}=0$. Given the particular form of $n^{\mu}$, we can recast the desired result as:
	
	\begin{align*}
	r=r_+ \implies g^{rr}=0 \numberthis \label{eq:demonstrandum}
	\end{align*}.

\begin{remark}
	We need to be careful, as the metric is singular when $r=r_+$. We shall develop the argument for general $r$, and take the limit as $r\rightarrow r_+$.
\end{remark}
 The following Lemma will be useful for computing $g^{\mu\nu}$ from $g_{\mu\nu}$.
\begin{lemma}\label{lemma:inverse}
	The inverse of a non-singular matrix  
	$A=\begin{pmatrix}
		\alpha & 0 & 0 & \epsilon \\
		0 & \beta & 0 & 0 \\
		0 & 0 & \gamma & 0 \\
		\epsilon & 0 & 0 & \delta
	\end{pmatrix}$ is given by  $A^{-1}=\frac{1}{(\alpha\delta-\epsilon^2)\beta\gamma}\begin{pmatrix}
	\beta\gamma\delta & 0 & 0 & -\epsilon\beta\gamma\\
	0 & \alpha\gamma\delta - \gamma \epsilon^2 & 0 & 0\\
	0 & 0 & \alpha\beta\delta - \beta \epsilon^2 & 0 \\
	-\epsilon\beta\gamma & 0 & 0 & \alpha\beta\gamma
	\end{pmatrix}
	$
\end{lemma}

\begin{proof}
	I used the well known result (e.g. \cite[Theorem 5.2]{finkbeiner1960}) that $A^{-1} = \frac{1}{|A|} adj(A)$, where $adj(A)$ is the matrix of cofactors of A, i.e. $|A|=\sum_{i} A_{ij}$ for fixed j. The Lemma may also be verified by multiplying A by its putative inverse.
\end{proof}	

Comparing with (\ref{eq:metric})
\begin{align*}
\alpha =g^{tt}=& \frac{\Delta-a^2 (\sin{\theta})^2}{\rho^2}\\
\beta=g^{rr}=& - \frac{\rho^2}{\Delta} \\
\gamma =g^{\theta\theta}=& -\rho^2\\
\delta=g^{\phi\phi}=& - \frac{A (\sin{\theta})^2}{\rho^2} \\ 
\epsilon =g^{\phi t}=g^{t\phi}=&\frac{2 \kappa M a}{\rho^2} r (\sin{\theta})^2 
\end{align*}

Lemma \ref{lemma:inverse} gives:
\begin{align*}
g^{rr}=& \frac{ \alpha\gamma\delta - \gamma \epsilon^2}{(\alpha\delta-\epsilon^2)\beta\gamma}\\
=& \frac{ \gamma(\alpha\delta -  \epsilon^2)}{(\alpha\delta-\epsilon^2)\beta\gamma}\\
=& \frac{ \alpha\delta -  \epsilon^2}{(\alpha\delta-\epsilon^2)\beta}\\
=& \frac{1}{\beta}\\
=& -\frac{\Delta}{\rho^2}
\end{align*}

Now the definition of $r_{\pm}$ is given by (\ref{eq:rpm}), and the definition of $\rho$ following (\ref{eq:metric}) $\implies \rho>0$, so, taking limits:

\begin{align*}
r\rightarrow r_{\pm} \implies& \Delta \rightarrow 0 \\
\implies& g^{rr} \rightarrow 0 \\
\implies& n^{\mu}n_{\mu} \rightarrow 0 \text{, which is what we wanted to prove--(\ref{eq:demonstrandum})}
\end{align*}
\end{proof}


\begin{thebibliography}{9}
	
	\bibitem{akhmedev2016}
	Emil T. Akhmedev,
	\emph{Lectures on General Theory of Relativity},
	2016,
	\url{https://arxiv.org/pdf/1601.04996v6.pdf}.
	
	\bibitem{finkbeiner1960}
	Daniel T Finkbeiner II.
	\emph{Introduction to Matrices and Linear Transformations},
	\url{http://store.doverpublications.com/048648159x.html}
	
\end{thebibliography}

\end{document}
