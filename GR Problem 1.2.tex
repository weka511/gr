\documentclass[11pt,a4paper]{article}
\usepackage[latin1]{inputenc}
\usepackage{amsmath}
\usepackage{amsfonts}
\usepackage{amssymb}
\usepackage{amsthm}
\usepackage{amscd}
\usepackage{graphicx}
\usepackage{tkz-euclide}
\usepackage{tikz}
\author{Simon Crase}
\title{General Relativity: Problem 2}
\begin{document}
\maketitle


\section{Rotating Coordinates}

\textit{Find the coordinate transformation and the stationary metric in the rotating reference system with the angular velocity $\omega$.}	

The figure below shows the xy coordinates for the Minkowski space (assuming rotation around the z axis with angular velocity $\omega$), the rotating x'y' coordinates, and polar coordinates r $\theta$ as alternative coordinates in the rotating coordinate system.

\begin{tikzpicture} \label{coordinates}
\draw (0,0) -- (8,0) node[anchor=north west] {x};
\draw (0,0) -- (0,8) node[anchor=south east] {y};
\draw[dashed] (0,0) -- (7,2) node[anchor=north west] {x'};
\draw[dashed] (0,0) -- (-2,7)  node[anchor=south east] {y'};
\draw (3,0) arc (0:17:3cm)  node[anchor=north west] {$\omega$t};
\draw[densely dotted] (0,0) -- (5,5)  node[anchor=south east] {r};
\draw (3.5,1) arc (17:53:3cm)  node[anchor=north west] {$\theta$};
\end{tikzpicture}

If a point has coordinates r and $\theta$ in the rotating coordinate system, its coordinates in the xy plane of Minkowski space will be as follows.
\begin{subequations}
	\begin{align}
	x &= r \cos(\theta+\omega t)\\
	y &= r \sin(\theta+\omega t)
	\end{align}
\end{subequations}

Using $dx = \frac{\partial x}{\partial r} dr + \frac{\partial x}{\partial \theta} d \theta + \frac{\partial x}{\partial t} dt$, we obtain:
\begin{subequations}
	\begin{align}
	dx &= \cos(\theta + \omega t) dr -r sin(\theta + \omega t) d\theta - \omega r sin(\theta + \omega t) dt\\
	dy &= \sin(\theta + \omega t) dr + r cos(\theta + \omega t) d\theta + \omega r cos(\theta + \omega t) dt
	\end{align}
\end{subequations}


\begin{align}
\begin{split}\label{eq:2}
dx^{2}+dy^{2} ={}& \cos^{2}(\theta + \omega t) dr^{2}+r^{2}\sin^{2}(\theta + \omega t)d\theta^{2}+\omega^{2}r^{2}\sin^{2}(\theta + \omega t)dt^{2} \\
&+ 2 \omega r^{2} \sin^{2}(\theta + \omega t)d \theta dt -2 \omega r \cos(\theta + \omega t)\sin(\theta + \omega t) dr dt\\
&-2 r \cos(\theta + \omega t)\sin(\theta + \omega t) dr d\theta +\sin^{2}(\theta + \omega t) dr^{2} \\
&+ r^{2}\cos^{2}(\theta + \omega t)d\theta^{2} + \omega^{2} r^{2} \cos^{2}(\theta + \omega t) dt^{2} \\
&+ 2 \omega r^{2} \cos^{2}(\theta + \omega t)d \theta dt +2 \omega r \cos(\theta + \omega t)\sin(\theta + \omega t) dr dt\\
&+2 r \cos(\theta + \omega t)\sin(\theta + \omega t) dr d\theta
\end{split}
\end{align}
Simplifying using basic trigonometric identities and cancelling, we get:
\begin{equation} \label{eq:2Dpolar}
dx^{2}+dy^{2} = dr^{2} + r^{2}d\theta^{2} + \omega^{2}r^{2}dt^{2}+2 \omega r^{2} d\theta dt
\end{equation}
In Minkowski space, the metric is:
\begin{align}
	ds^{2}&=dt^{2}-dx^{2}-dy^{2}-dz^{2}\\
	&=(1-\omega^{2}r^{2})dt^{2}-dr^{2} - r^{2}d\theta^{2} -2 \omega r^{2} d\theta dt
\end{align}
upon substituting \ref{eq:2Dpolar}
\end{document}